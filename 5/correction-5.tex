\documentclass{article}
\errorcontextlines 10000

\makeatletter

\usepackage{fontspec}
\usepackage{xunicode}
\usepackage{xltxtra}

\usepackage{fullpage}
\usepackage{mleftright}

\usepackage{hyperref}

\hypersetup{%
	linktocpage  = true, %page number is the link (not title)
	colorlinks   = true, %Colours links instead of ugly boxes
	urlcolor     = darkblue, %Colour for external hyperlinks
	linkcolor    = darkblue, %Colour of internal links
	citecolor    = darkblue   %Colour of citations
}

%\hypersetup{
%	linktocpage  = true, %page number is the link (not title)
%	colorlinks   = true, %Colours links instead of ugly boxes
%	allcolors = bookColor,
%	hidelinks = true
%}

% good looking urls
\urlstyle{same}

\newcommand{\theoremname}{Theorem}
\newcommand{\corollaryname}{Corollary}
\newcommand{\problemname}{Problem}
\newcommand{\conjecturename}{Conjecture}
\newcommand{\definitionname}{Definition}

% we use this for our references as well
\let\sref\ref
\AtBeginDocument{\renewcommand{\ref}[1]{\mbox{\autoref{#1}}}}
\usepackage[nameinlink]{cleveref}

% define colors %
\usepackage[table]{xcolor}

%\usepackage[table,dvipdfx,cmyk]{xcolor}
%\definecolor{bookColor}{cmyk}{0 ,0 ,0 ,1}
%\color{bookColor}

\definecolor{lightgray}{rgb}{0.9,0.9,0.9}
\definecolor{darkgray}{rgb}{0.4,0.4,0.4}
\definecolor{purple}{rgb}{0.65, 0.12, 0.82}
\definecolor{darkblue}{rgb}{0.02, 0.17, 0.40}
\usepackage[stable]{footmisc}
\usepackage{graphicx}
\usepackage{caption}
\usepackage{subcaption}
\usepackage{tikz}
\usepackage{url}
\usepackage{amsmath,amsthm,amsfonts,mathtools}
\usepackage{xfrac}
\newtheorem{theorem}{\theoremname}
\newtheorem{corollary}{\corollaryname}
\newtheorem{problem}{\problemname}
\newtheorem{conjecture}{\conjecturename}
\newtheorem{definition}{\definitionname}

% nicely spaced operators
\DeclareMathOperator{\Exists}{\exists}
\DeclareMathOperator{\Forall}{\forall}

\DeclarePairedDelimiter{\ceil}{\lceil}{\rceil}
\newcommand{\card}[1]{|#1|}

\newcommand{\st}{\colon\,}
\newcommand{\TM}{TM}
\newcommand{\Atm}{A\textsubscript{TM}}
\newcommand{\HALTtm}{HALT\textsubscript{TM}}

\newcommand\N{\mathbb{N}_1}

\newcommand*\circled[1]{\tikz[baseline=(char.base)]{
            \node[shape=circle,draw,inner sep=2pt] (char) {#1};}}



% TURING MACHINE DESCRIPTIONS
\usepackage{changepage}
\newenvironment{steps}%
{%
\vspace{0.25cm}%
\begin{adjustwidth}{0.3cm}{0cm}%
\begin{description}%
}
{%
\end{description}%
\end{adjustwidth}%
\vspace{0.1cm}%
}

\newcounter{TMachine}[section]
\renewcommand{\theTMachine}{\thesection.\arabic{TMachine}}%
\newenvironment{TMachine}[1]
  {\refstepcounter{TMachine}%
   \par%
   \vspace{.5\baselineskip\@plus.2\baselineskip\@minus.2\baselineskip}% Space above
   \noindent{#1}%
   \begin{steps}}%\begin{TMachine}
  {\end{steps}%
\vspace{.5\baselineskip\@plus.2\baselineskip\@minus.2\baselineskip}% Space below
}% \end{TMachine}

\newcommand{\accept}{\emph{accept}}
\newcommand{\reject}{\emph{reject}}

\makeatother

\title{Computability and Complexity:\\Exercise Session 5 (2015-10-23)}
\author{Aurélien Ooms\footnote{aureooms@ulb.ac.be}}
\date{\today}

\begin{document}
\maketitle
\tableofcontents

\section{3SAT is polynomial time reducible to CLIQUE\footnote{%
Theorem 7.32 from the reference book: Sipser M.,
\emph{Introduction to the Theory of Computation}, 3rd edition (2013).}}

\subsection{Introduction}

Let us recall the big picture. We have shown in the first part of the course,
the part about computability, that not all problems can be solved by means of
Turing machines. We made the distinction between decidable and undecidable
problems, the former being the ones we can solve with algorithms, that is,
procedures that eventually terminate.

The second part, that is, the part about complexity, focuses on the resources
it takes to solve decidable problems. We already introduced the classes P and
NP. NP is the class of problems for which there exists a nondeterministic
polynomial-time algorithm. An equivalent definition is to consider the
languages for which all words have at least one corresponding certificate of
appartenance of polynomial size that can be verified using a polynomial
algorithm called a verifier. P is the class of problems for which there exists
a deterministic polynomial-time algorithm. By definition P is contained in NP.
However, we do not know whether P is different from NP.

Later on in the lectures, we will define the notion of NP-completeness and show
that, while we (today) cannot answer the question ``P vs. NP'', we can identify
a set of problems, the NP-complete problems, that capture the complexity of all
the problems in NP. This kind of result relies heavily on the notion of
reduction.

\subsection{Definitions}

We already saw that the reducibility of a problem to
another need not rely on similarity between the two problems. It is
possible to prove reducibility results between two apriori unrelated problems.
The goal of this exercise is to show you a first reduction of an NP-complete
problem to a problem in NP.

Let us begin with a few definitions
\begin{definition}
	A boolean formula \(\Phi\) over \(n\) variables \(x_1,x_2,\ldots,x_n\) is in
	conjunctive normal form (CNF) if \(\Phi\) can be
	written as a conjunction of \(m\) clauses \(c_1 \land c_2 \land \cdots \land
	c_m\) where each clause \(c_j\) can be written as a disjunction of literals
	\((l_1^j \lor l_2^j \lor \cdots \lor l_{k_j}^j)\) where each literal is
	either a variable \(x_i\) or the negation of a variable \(\lnot x_i\).
\end{definition}

The language CNF-SAT is defined as
\begin{definition}
	CNF-SAT \(\{ \langle \Phi \rangle \st \Phi \text{ is in CNF and there exists an assignment \(\in
		\{0,1\}^n\) on the
		variables \(x_1,x_2,\ldots,x_n\) that makes the formula evaluate to
		true}\}\)
\end{definition}

The language 3SAT is a subset of CNF-SAT
\begin{definition}
	3SAT \(\{ \langle \Phi \rangle \st \Phi \in \text{ CNF-SAT and has exactly 3 literals per clause}\}\)
\end{definition}

The language CLIQUE is the following
\begin{definition}
	CLIQUE \(\{ \langle G , k \rangle \st \text{ \(G\) is a graph with a clique
		of size \(k\)\}\)
\end{definition}

We now prove the following
\begin{theorem}
	3SAT \(\le_P\) CLIQUE
\end{theorem}

\begin{proof}
\end{proof}

What we just did is called a Karp reduction. Given two problems \(A\)
and \(B\), an algorithm for problem \(B\), and a polynomial
procedure to translate any
instance \(a\) of \(A\) into an instance \(b\) of \(B\) in such a way that the
answer to the decision problem \(a\) is ``yes'' if and only if the answer to
the decision problem \(b\) is ``yes'', we can decide any instance of \(A\) using
following algorithm
\begin{TMachine}{On input \(\langle a \rangle\):}
\item[1.] Translate instance \(a \in A\) to the corresponding instance \(b \in
	B\) in polynomial time.
\item[2.] Answer the same as the algorithm for \(B\) run on \(\langle b \rangle\).
\end{TMachine}
Moreover, if the algorithm for \(B\) runs in polynomial time so does this
algorithm.

\subsection{CLIQUE is NP-complete}

During the next lecture, we will define the notion of NP-completeness
\begin{definition}
	A language \(B\) is NP-complete if it satisfies two conditions:
	\begin{enumerate}
		\item \(B\) is in NP, and
		\item every \(A\) in NP is polynomial time reducible to \(B\).
	\end{enumerate}
\end{definition}
and see that
\begin{theorem}
	CNF-SAT is NP-complete
\end{theorem}
and that
\begin{theorem}
	CNF-SAT \(\le_P\) 3SAT.
\end{theorem}
Those two theorems together with the one we just proved and the fact that
CLIQUE is in NP\footnote{Left as exercise for the reader}
imply the following corollary
\begin{corollary}
	CLIQUE is NP-complete.
\end{corollary}
\end{document}
