\documentclass{article}
\errorcontextlines 10000

\makeatletter

\usepackage{fontspec}
\usepackage{xunicode}
\usepackage{xltxtra}

\usepackage{fullpage}
\usepackage{mleftright}

\usepackage{hyperref}

\hypersetup{%
	linktocpage  = true, %page number is the link (not title)
	colorlinks   = true, %Colours links instead of ugly boxes
	urlcolor     = darkblue, %Colour for external hyperlinks
	linkcolor    = darkblue, %Colour of internal links
	citecolor    = darkblue   %Colour of citations
}

%\hypersetup{
%	linktocpage  = true, %page number is the link (not title)
%	colorlinks   = true, %Colours links instead of ugly boxes
%	allcolors = bookColor,
%	hidelinks = true
%}

% good looking urls
\urlstyle{same}

\newcommand{\theoremname}{Theorem}
\newcommand{\corollaryname}{Corollary}
\newcommand{\problemname}{Problem}
\newcommand{\conjecturename}{Conjecture}
\newcommand{\definitionname}{Definition}

% we use this for our references as well
\let\sref\ref
\AtBeginDocument{\renewcommand{\ref}[1]{\mbox{\autoref{#1}}}}
\usepackage[nameinlink]{cleveref}

% define colors %
\usepackage[table]{xcolor}

%\usepackage[table,dvipdfx,cmyk]{xcolor}
%\definecolor{bookColor}{cmyk}{0 ,0 ,0 ,1}
%\color{bookColor}

\definecolor{lightgray}{rgb}{0.9,0.9,0.9}
\definecolor{darkgray}{rgb}{0.4,0.4,0.4}
\definecolor{purple}{rgb}{0.65, 0.12, 0.82}
\definecolor{darkblue}{rgb}{0.02, 0.17, 0.40}
\usepackage[stable]{footmisc}
\usepackage{graphicx}
\usepackage{caption}
\usepackage{subcaption}
\usepackage{tikz}
\usepackage{url}
\usepackage{amsmath,amsthm,amsfonts}
\newtheorem{theorem}{\theoremname}
\newtheorem{corollary}{\corollaryname}
\newtheorem{problem}{\problemname}
\newtheorem{conjecture}{\conjecturename}
\newtheorem{definition}{\definitionname}

% nicely spaced operators
\DeclareMathOperator{\Exists}{\exists}
\DeclareMathOperator{\Forall}{\forall}

\newcommand{\card}[1]{|#1|}

\newcommand{\st}{\colon\,}
\newcommand{\TM}{TM}
\newcommand{\Atm}{A\textsubscript{TM}}
\newcommand{\HALTtm}{HALT\textsubscript{TM}}

\newcommand\N{\mathbb{N}_1}

\newcommand*\circled[1]{\tikz[baseline=(char.base)]{
            \node[shape=circle,draw,inner sep=2pt] (char) {#1};}}



% TURING MACHINE DESCRIPTIONS
\usepackage{changepage}
\newenvironment{steps}%
{%
\vspace{0.25cm}%
\begin{adjustwidth}{0.3cm}{0cm}%
\begin{description}%
}
{%
\end{description}%
\end{adjustwidth}%
\vspace{0.1cm}%
}

\newcounter{TMachine}[section]
\renewcommand{\theTMachine}{\thesection.\arabic{TMachine}}%
\newenvironment{TMachine}[1]
  {\refstepcounter{TMachine}%
   \par%
   \vspace{.5\baselineskip\@plus.2\baselineskip\@minus.2\baselineskip}% Space above
   \noindent{#1}%
   \begin{steps}}%\begin{TMachine}
  {\end{steps}%
\vspace{.5\baselineskip\@plus.2\baselineskip\@minus.2\baselineskip}% Space below
}% \end{TMachine}

\newcommand{\accept}{\emph{accept}}
\newcommand{\reject}{\emph{reject}}

\makeatother

\title{Computability and Complexity:\\Exercise Session 5 (2017-11-08)}
\author{Aurélien Ooms\footnote{aureooms@ulb.ac.be}}
\date{\today}

\begin{document}
\maketitle
\tableofcontents

\section{Exercise 5.9}

We showed during a previous lecture (2017-10-23) that given some non-trivial property
\(P\) the problem of deciding whether the language \(L(M)\) of some TM \(M\)
has property \(P\), that is, \(L(M) \in P\), is undecidable. This result is
Rice's theorem. To solve this exercise, we just have to show
that\footnote{\(w = w_1 w_2 \ldots w_n \iff w^{\mathcal{R}} = w_{n} w_{n-1}\ldots w_{1}\)}
\(P = \{A \st w \in A \iff
w^{\mathcal{R}} \in A \}\) is indeed a non-trivial\footnote{Non-trivial means
that \(P \neq \emptyset \land P \neq \mathcal{P}(\Sigma^*)\), where
\(\mathcal{P}(S)\) is the set of all subsets of \(S\). \(\mathcal{P}\) is
called the powerset operator.} property. For
example, \(\{\texttt{01}\} \not\in P\) so \(P \neq \mathcal{P}(\Sigma^*)\) and
\(\{\texttt{1}\} \in P\) so \(P \neq \emptyset\).

Alternatively, we can produce the following proof reducing \Atm{} to our problem.
Let us assume \(T = \{\langle M \rangle \st M \text{ is a \TM{} that accepts
\(w^{\mathcal{R}}\) whenever it accepts \(w\)}\}\) is decidable. That means
there exists a decider \(H\) for language \(T\). Given some input \(\langle
M,w\rangle\) to test against the language \Atm{}, build the following
\TM{}:

\begin{TMachine}{\(D_{\langle M,w\rangle} =\) on input \(x\):}
\item[1.] If \(x = \texttt{01}\): \accept.
\item[2.] Otherwise run \(M\) on \(w\). If \(M\) accepts, \accept. Otherwise
		\reject.
\end{TMachine}
Note that \(L(D) = \Sigma^* \in P\) if and only if \(M\) accepts \(w\) and
\(L(D) = \{\texttt{01}\} \not\in P\) if and only if \(M\) does not accept
\(w\). Hence, for any input \(\langle M,w\rangle\) we can decide whether \(M\)
accepts \(w\) by running the decider \(H\) on \(\langle D_{\langle M,w\rangle}
\rangle\), a contradiction with respect to the theorem about the undecidability
of \Atm{} we proved during the lectures.

\section{CLIQUE, VERTEX-COVER, and INDEPENDENT-SET}

Let us first define the concepts of clique and independent set in a graph.
Given a graph, a clique is a subset of its vertices such that \textbf{each} pair in this
subset is connected by an edge.

\begin{definition}[Clique]
  Let $G=(V,E)$, $S\subseteq V$. The set of vertices $S$ is a clique of $G$
  if and only if for all $(u,v) \in S^2$ with $u\ne v$, $\{\,u,v\,\} \in E$.
\end{definition}

Given a graph, an independent set is a subset of its vertices such that
\textbf{no} pair in this
subset is connected by an edge.
\begin{definition}[Independent set]
  Let $G=(V,E)$, $S\subseteq V$. The set of vertices $S$ is an independent set
  of $G$ if and only if for all $(u,v) \in S^2$ with $u\ne v$, $\{\,u,v\,\} \not\in E$.
\end{definition}

We show that, as you may already have noticed, those two concepts are
closely related. To achieve this goal, let us define the complement of a graph
$G$, $\bar{G}$, that has the same vertex set as $G$, but whose edge set
contains only the edges that are missing in $G$:
\begin{definition}[Complement of a graph]
  Let $G = (V,E)$, define $\bar{E}=\{\,\{\,u,v\,\} \st (u,v) \in S^2,
  u \ne v, \{\,u,v\,\} \not\in E\,\}$ and $\bar{G} = (V,\bar{E})$.
\end{definition}

We are now ready to prove our first result:
\begin{theorem}\label{is-cl}
  Let $G = (V,E)$, $S \subseteq V$. The set $S$ is an independent set in $G$ if
  and only if $S$ is a clique in $\bar{G}$.
\end{theorem}

\begin{proof}
  Let us prove the two directions of the implication.
  \paragraph{($\implies$)}%
  Let $S$ be an independent set in $G$.
  For any $(u,v) \in S^2, u \ne v$, by definition of an independent set,
  $\{\,u,v\,\} \not\in E$, hence, $\{\,u,v\,\} \in \bar{E}$. Hence, by
  definition of a clique, $S$ is a clique in $\bar{G}$.

  \paragraph{($\impliedby$)}%
  Conversely, let $S$ be a clique in $\bar{G}$.
  For any $(u,v) \in S^2, u \ne v$, by definition of a clique,
  $\{\,u,v\,\} \in \bar{E}$, hence, $\{\,u,v\,\} \not\in E$. Hence, by
  definition of an independent set, $S$ is an independent set in $G$.
\end{proof}

We define a third concept, the vertex cover. Given a graph, a vertex cover in
this graph covers all its edges by containing at least one endpoint of each.
\begin{definition}[Vertex cover]
  Let $G=(V,E)$, $S\subseteq V$. The set of vertices $S$ is a vertex cover of
  $G$ if and only if for all $\{u,v\} \in E$, $u \in S \lor v \in S$.
\end{definition}

Again, although maybe less obvious, the concepts of vertex cover and
independent set are closely related.
\begin{theorem}\label{is-vc}
  Let $G = (V,E)$, $S \subseteq V$. The set $S$ is an independent set in $G$ if
  and only if $T = V \setminus S$ is a vertex cover in $G$.
\end{theorem}

\begin{proof}
  Let us prove the two directions of the implication.
  \paragraph{($\implies$)}%
  Let $S$ be an independent set in $G$ and define $T = V \setminus S$.
  For any $\{\,u,v\,\} \in E$, by definition of an independent set, at most one
  of $u$ and $v$ belongs to $S$, hence, at least one of $u$ and $v$ belongs to
  $T$. Hence, by definition of a vertex cover, $T$ is a vertex cover in $G$.

  \paragraph{($\impliedby$)}%
  Conversely, let $T$ be a vertex cover in $G$ and define $S = V \setminus T$.
  For any $\{\,u,v\,\} \in E$, by definition of a vertex cover, at least one
  of $u$ and $v$ belongs to $T$, hence, at most one of $u$ and $v$ belongs to
  $S$. Hence, by definition of an independent set, $S$ is an independent set in
  $G$.
\end{proof}

As an exercise, you can show the following:
\begin{corollary}\label{vc-cl}
  Let $G = (V,E)$, $S \subseteq V$. The set $S$ is an clique in $G$ if
  and only if $T = V \setminus S$ is a vertex cover in $\bar{G}$.
\end{corollary}

We are interested in the complexities of the problems of deciding the existence
of those substructures in arbitrary graphs. We define the following languages:
\begin{definition}
  CLIQUE $= \{\langle G,k \rangle \st \text{$G$ is a graph and contains a clique of size $k$.} \}$
\end{definition}

\begin{definition}
  INDEPENDENT-SET $= \{\langle G,k \rangle \st \text{$G$ is a graph and contains an independent set of size $k$.} \}$
\end{definition}

\begin{definition}
  VERTEX-COVER $= \{\langle G,k \rangle \st \text{$G$ is a graph and contains a vertex-cover of size $k$.} \}$
\end{definition}

During a previous lecture, you proved that CLIQUE is complete for the
complexity class NP. This means that (1) CLIQUE is in NP, that is, it is as
``easy'' as all other problems in NP, and (2), CLIQUE is NP-hard,
that is, it is as ``hard'' as all other problems in NP.
\begin{theorem}\label{cl:npc}
CLIQUE is NP-complete.
\end{theorem}
\begin{proof}
  See lecture notes.
\end{proof}

Building on~\ref{is-cl}~and~\ref{cl:npc}, we can prove the following:
\begin{theorem}\label{is:npc}
INDEPENDENT-SET is NP-complete.
\end{theorem}

\begin{proof}
  We must prove that INDEPENDENT-SET is both in NP and NP-hard.
  \paragraph{$\in$ NP}
  By~\ref{is-cl}, $\langle G,k \rangle \in$ INDEPENDENT-SET if (and only if)
  $\langle \bar{G},k\rangle \in$ CLIQUE.
  In other words, deciding if a graph $G$ contains an independent set of size $k$ is
  equivalent to deciding if the complement of $G$, $\bar{G}$, contains a
  clique of size $k$. The reduction takes quadratic time in the worst
  case where $G$ contains no edge and $\bar{G}$ contains $\binom{\card{V}^2}{2}$.
  By~\ref{cl:npc}, CLIQUE is in NP, that is, ``easy''. Since
  we can solve an instance of INDEPENDENT-SET by solving a polynomial size instance of
  CLIQUE then INDEPENDENT-SET must be as ``easy'' as CLIQUE, that
  is, INDEPENDENT-SET is in NP.

  \paragraph{NP-hard}
  By~\ref{is-cl}, $\langle G,k \rangle \in$ CLIQUE if (and only if)
  $\langle \bar{G},k\rangle \in$ INDEPENDENT-SET.
  In other words, deciding if a graph $G$ contains a clique of size $k$ is
  equivalent to deciding if the complement of $G$, $\bar{G}$, contains an
  independent set of size $k$. The reduction takes quadratic time in the worst
  case where $G$ contains no edge and $\bar{G}$ contains $\binom{\card{V}^2}{2}$.
  By~\ref{cl:npc}, CLIQUE is NP-hard, that is, ``hard''. Since
  we can solve an instance of CLIQUE by solving a polynomial size instance of
  INDEPENDENT-SET then INDEPENDENT-SET must be as ``hard'' as CLIQUE, that
  is, INDEPENDENT-SET is NP-hard.
\end{proof}

Building on~\ref{is-vc}~and~\ref{is:npc}, we can prove the following:
\begin{theorem}\label{vc:npc}
VERTEX-COVER is NP-complete.
\end{theorem}

\begin{proof}
  We must prove that VERTEX-COVER is both in NP and NP-hard.
  \paragraph{$\in$ NP}
  By~\ref{is-vc}, $\langle G,k \rangle \in$ VERTEX-COVER if (and only if)
  $\langle G,\card{V}-k\rangle \in$ INDEPENDENT-SET.
  In other words, deciding if a graph $G$ contains a vertex cover of size $k$ is
  equivalent to deciding if $G$ contains an
  independent set of size $\card{V}-k$. The reduction takes linear time as $G$ stays the
  same and $\card{V}-k$ has bit-size $\lceil\log \card{V} - k\rceil \le
  \max(\log\card{V},\log k) + 1$.
  By~\ref{is:npc}, INDEPENDENT-SET is in NP, that is, ``easy''. Since
  we can solve an instance of VERTEX-COVER by solving a polynomial size instance of
  INDEPENDENT-SET then VERTEX-COVER must be as ``easy'' as INDEPENDENT-SET, that
  is, VERTEX-COVER is in NP.

  \paragraph{NP-hard}
  By~\ref{is-vc}, $\langle G,k \rangle \in$ INDEPENDENT-SET if (and only if)
  $\langle G,\card{V}-k\rangle \in$ VERTEX-COVER.
  In other words, deciding if a graph $G$ contains an independent set of size $k$ is
  equivalent to deciding if $G$ contains a
  vertex cover of size $\card{V}-k$. The reduction takes linear time as $G$ stays the
  same and $\card{V}-k$ has bit-size $\lceil\log \card{V} - k\rceil \le
  \max(\log\card{V},\log k) + 1$.
  By~\ref{is:npc}, INDEPENDENT-SET is NP-hard, that is, ``hard''. Since
  we can solve an instance of INDEPENDENT-SET by solving a polynomial size instance of
  VERTEX-COVER then VERTEX-COVER must be as ``hard'' as INDEPENDENT-SET, that
  is, VERTEX-COVER is NP-hard.
\end{proof}

\end{document}
