\documentclass{article}
\errorcontextlines 10000

\makeatletter

\usepackage{fontspec}
\usepackage{xunicode}
\usepackage{xltxtra}

\usepackage{fullpage}
\usepackage{mleftright}
\usepackage{hhline}

\usepackage{hyperref}

\hypersetup{%
	linktocpage  = true, %page number is the link (not title)
	colorlinks   = true, %Colours links instead of ugly boxes
	urlcolor     = darkblue, %Colour for external hyperlinks
	linkcolor    = darkblue, %Colour of internal links
	citecolor    = darkblue   %Colour of citations
}

%\hypersetup{
%	linktocpage  = true, %page number is the link (not title)
%	colorlinks   = true, %Colours links instead of ugly boxes
%	allcolors = bookColor,
%	hidelinks = true
%}

% good looking urls
\urlstyle{same}

\newcommand{\theoremname}{Theorem}
\newcommand{\corollaryname}{Corollary}
\newcommand{\problemname}{Problem}
\newcommand{\conjecturename}{Conjecture}
\newcommand{\definitionname}{Definition}

% we use this for our references as well
\let\sref\ref
\AtBeginDocument{\renewcommand{\ref}[1]{\mbox{\autoref{#1}}}}
\usepackage[nameinlink]{cleveref}

% define colors %
\usepackage[table]{xcolor}

%\usepackage[table,dvipdfx,cmyk]{xcolor}
%\definecolor{bookColor}{cmyk}{0 ,0 ,0 ,1}
%\color{bookColor}

\definecolor{lightgray}{rgb}{0.9,0.9,0.9}
\definecolor{darkgray}{rgb}{0.4,0.4,0.4}
\definecolor{purple}{rgb}{0.65, 0.12, 0.82}
\definecolor{darkblue}{rgb}{0.02, 0.17, 0.40}
\usepackage[stable]{footmisc}
\usepackage{graphicx}
\usepackage{caption}
\usepackage{subcaption}
\usepackage{tikz}
\usepackage{url}
\usepackage{amsmath,amsthm,amsfonts,mathtools}
\usepackage{xfrac}


\newtheorem{theorem}{\theoremname}
\newtheorem{corollary}{\corollaryname}
\newtheorem{problem}{\problemname}
\newtheorem{conjecture}{\conjecturename}
\newtheorem{definition}{\definitionname}

% nicely spaced operators
\DeclareMathOperator{\Exists}{\exists}
\DeclareMathOperator{\Forall}{\forall}

\DeclarePairedDelimiter{\ceil}{\lceil}{\rceil}
\newcommand{\card}[1]{|#1|}

\newcommand{\st}{\colon\,}
\newcommand{\TM}{TM}
\newcommand{\Atm}{A\textsubscript{TM}}
\newcommand{\HALTtm}{HALT\textsubscript{TM}}

\newcommand{\N}{\mathbb{N}}
\newcommand{\nth}[1]{#1^{\text{th}}}

\newcommand*\circled[1]{\tikz[baseline=(char.base)]{
            \node[shape=circle,draw,inner sep=2pt] (char) {#1};}}



% TURING MACHINE DESCRIPTIONS
\usepackage{changepage}
\newenvironment{steps}%
{%
\vspace{0.25cm}%
\begin{adjustwidth}{0.3cm}{0cm}%
\begin{description}%
}
{%
\end{description}%
\end{adjustwidth}%
\vspace{0.1cm}%
}

\newcounter{TMachine}[section]
\renewcommand{\theTMachine}{\thesection.\arabic{TMachine}}%
\newenvironment{TMachine}[1]
  {\refstepcounter{TMachine}%
   \par%
   \vspace{.5\baselineskip\@plus.2\baselineskip\@minus.2\baselineskip}% Space above
   \noindent{#1}%
   \begin{steps}}%\begin{TMachine}
  {\end{steps}%
\vspace{.5\baselineskip\@plus.2\baselineskip\@minus.2\baselineskip}% Space below
}% \end{TMachine}

\newcommand{\accept}{\emph{accept}}
\newcommand{\reject}{\emph{reject}}

\makeatother

\title{Computability and Complexity:\\Exercise Session 8 (2015-12-04)}
\author{Aurélien Ooms\footnote{aureooms@ulb.ac.be}}
\date{\today}

\begin{document}
\maketitle
\tableofcontents

\section{\texorpdfstring{TQBF is in PSPACE\footnote{%
First part in Theorem 8.9 from the reference book: Sipser M.,
\emph{Introduction to the Theory of Computation}, 3rd edition
(2013).}}{TQBF is in PSPACE}}

As a reminder
\begin{definition}
	SPACE\((f(n))\) \(= \{L \st L \) is a language decided by an \(O(f(n))\) space
	deterministic Turing machine.\(\}\)
\end{definition}
\begin{definition}
	PSPACE \(= \bigcup_k\) SPACE\((n^k)\)
\end{definition}
\begin{definition}
	TQBF \(= \{ \langle \phi \rangle \st \phi\) is a true fully quantified
	boolean formula.\(\}\)
\end{definition}

\begin{theorem}
	TQBF \(\in\) PSPACE
\end{theorem}

\begin{proof}
Obviously, the following recursive algorithm decides TQBF
\begin{TMachine}{\(T =\) ``On input \(\langle \varphi \rangle\), a fully quantified Boolean formula:}
\item[1.] If \(\varphi\) contains no quantifiers, then it is an expression with only
constants, so evaluate \(\varphi\) and accept if it is true; otherwise, reject.
\item[2.] If \(\varphi\) equals \(\exists x~\phi\), recursively call \(T\) on
	\(\phi\), first with \(0\) substituted
	for \(x\) and then with \(1\) substituted for \(x\). If either result is accept,
then \accept; otherwise, \reject.
\item[3.] If \(\varphi\) equals \(\forall x~\phi\), recursively call \(T\) on
	\(\phi\), first with \(0\) substituted
	for \(x\) and then with \(1\) substituted for \(x\). If both results are
	accept, then \accept; otherwise, \reject.''
\end{TMachine}

The depth of the recursion is \(m\), the number of variables.
Note that \(m = O(n)\), where \(n\) is the size of the input formula.
At each level of
the recursion we only need to store the value of one variable and so the space
used is \(O(n)\). Hence, this algorithm decides TQBF using linear space.

\end{proof}

\section{FORMULA-GAME = TQBF}

See the complete subsection \emph{``WINNING STRATEGIES FOR GAMES''} in the
texbook, and in particular Theorem 8.11 together with its proof.
Pages 313 to 315 in the second edition. Pages 341 to 343 in the third
edition.

\end{document}
