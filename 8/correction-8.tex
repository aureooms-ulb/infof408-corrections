\documentclass{article}
\errorcontextlines 10000

\makeatletter

\usepackage{fontspec}
\usepackage{xunicode}
\usepackage{xltxtra}

\usepackage{fullpage}
\usepackage{mleftright}
\usepackage{hhline}

\usepackage{hyperref}

\hypersetup{%
	linktocpage  = true, %page number is the link (not title)
	colorlinks   = true, %Colours links instead of ugly boxes
	urlcolor     = darkblue, %Colour for external hyperlinks
	linkcolor    = darkblue, %Colour of internal links
	citecolor    = darkblue   %Colour of citations
}

%\hypersetup{
%	linktocpage  = true, %page number is the link (not title)
%	colorlinks   = true, %Colours links instead of ugly boxes
%	allcolors = bookColor,
%	hidelinks = true
%}

% good looking urls
\urlstyle{same}

\newcommand{\theoremname}{Theorem}
\newcommand{\corollaryname}{Corollary}
\newcommand{\problemname}{Problem}
\newcommand{\conjecturename}{Conjecture}
\newcommand{\definitionname}{Definition}

% we use this for our references as well
\let\sref\ref
\AtBeginDocument{\renewcommand{\ref}[1]{\mbox{\autoref{#1}}}}
\usepackage[nameinlink]{cleveref}

% define colors %
\usepackage[table]{xcolor}

%\usepackage[table,dvipdfx,cmyk]{xcolor}
%\definecolor{bookColor}{cmyk}{0 ,0 ,0 ,1}
%\color{bookColor}

\definecolor{lightgray}{rgb}{0.9,0.9,0.9}
\definecolor{darkgray}{rgb}{0.4,0.4,0.4}
\definecolor{purple}{rgb}{0.65, 0.12, 0.82}
\definecolor{darkblue}{rgb}{0.02, 0.17, 0.40}
\usepackage[stable]{footmisc}
\usepackage{graphicx}
\usepackage{caption}
\usepackage{subcaption}
\usepackage{tikz}
\usepackage{url}
\usepackage{amsmath,amsthm,amsfonts,mathtools}
\usepackage{xfrac}


\newtheorem{theorem}{\theoremname}
\newtheorem{corollary}{\corollaryname}
\newtheorem{problem}{\problemname}
\newtheorem{conjecture}{\conjecturename}
\newtheorem{definition}{\definitionname}

% nicely spaced operators
\DeclareMathOperator{\Exists}{\exists}
\DeclareMathOperator{\Forall}{\forall}

\DeclarePairedDelimiter{\ceil}{\lceil}{\rceil}
\newcommand{\card}[1]{|#1|}

\newcommand{\st}{\colon\,}
\newcommand{\TM}{TM}
\newcommand{\Atm}{A\textsubscript{TM}}
\newcommand{\HALTtm}{HALT\textsubscript{TM}}

\newcommand{\N}{\mathbb{N}}
\newcommand{\nth}[1]{#1^{\text{th}}}

\newcommand*\circled[1]{\tikz[baseline=(char.base)]{
            \node[shape=circle,draw,inner sep=2pt] (char) {#1};}}



% TURING MACHINE DESCRIPTIONS
\usepackage{changepage}
\newenvironment{steps}%
{%
\vspace{0.25cm}%
\begin{adjustwidth}{0.3cm}{0cm}%
\begin{description}%
}
{%
\end{description}%
\end{adjustwidth}%
\vspace{0.1cm}%
}

\newcounter{TMachine}[section]
\renewcommand{\theTMachine}{\thesection.\arabic{TMachine}}%
\newenvironment{TMachine}[1]
  {\refstepcounter{TMachine}%
   \par%
   \vspace{.5\baselineskip\@plus.2\baselineskip\@minus.2\baselineskip}% Space above
   \noindent{#1}%
   \begin{steps}}%\begin{TMachine}
  {\end{steps}%
\vspace{.5\baselineskip\@plus.2\baselineskip\@minus.2\baselineskip}% Space below
}% \end{TMachine}

\newcommand{\accept}{\emph{accept}}
\newcommand{\reject}{\emph{reject}}

\makeatother

\title{Computability and Complexity:\\Exercise Session 8 (2016-11-23)}
\author{Aurélien Ooms\footnote{aureooms@ulb.ac.be}}
\date{\today}

\begin{document}
\maketitle
\tableofcontents

\section{CLIQUE, VERTEX-COVER, and INDEPENDENT-SET}

\begin{definition}[Independent set]
  Let $G=(V,E)$, $S\subseteq V$. The set of vertices $S$ is an independent set
  of $G$ if and only if for all $(u,v) \in S^2$ with $u\ne v$, $\{\,u,v\,\} \not\in E$.
\end{definition}

\begin{definition}[Clique]
  Let $G=(V,E)$, $S\subseteq V$. The set of vertices $S$ is a clique of $G$
  if and only if for all $(u,v) \in S^2$ with $u\ne v$, $\{\,u,v\,\} \in E$.
\end{definition}

\begin{definition}[Vertex cover]
  Let $G=(V,E)$, $S\subseteq V$. The set of vertices $S$ is a vertex cover of
  $G$ if and only if for all $\{u,v\} \in E$, $u \in S \lor v \in S$.
\end{definition}

\begin{definition}[Complement of a graph]
  Let $G = (V,E)$, define $\bar{E}=\{\,\{\,u,v\,\} \st (u,v) \in S^2,
  u \ne v, \{\,u,v\,\} \not\in E\,\}$ and $\bar{G} = (V,\bar{E})$.
\end{definition}

\begin{theorem}
  Let $G = (V,E)$, $S \subseteq V$. The set $S$ is an independent set in $G$ if
  and only if $S$ is a clique in $\bar{G}$.
\end{theorem}

\begin{proof}
  Let us prove the two directions of the implication.
  \paragraph{($\implies$)}%
  Let $S$ be an independent set in $G$.
  For any $\{\,u,v\,\} \in E$, by definition of an independent set, at most one
  of $u$ and $v$ belongs to $S$, hence, at least one of $u$ and $v$ belongs to
  $T$. Hence, by definition of a vertex cover, $T$ is a vertex cover.

  \paragraph{($\impliedby$)}%
  Let $T$ be a vertex cover in $G$.
  For any $\{\,u,v\,\} \in E$, by definition of a vertex cover, at least one
  of $u$ and $v$ belongs to $T$, hence, at most one of $u$ and $v$ belongs to
  $S$. Hence, by definition of an independent set, $S$ is an independent set.
\end{proof}

\begin{theorem}
  Let $G = (V,E)$, $S \subseteq V$. The set $S$ is an independent set in $G$ if
  and only if $T = V \setminus S$ is a vertex cover in $G$.
\end{theorem}

\begin{proof}
  Let us prove the two directions of the implication.
  \paragraph{($\implies$)}%
  Let $S$ be an independent set in $G$ and define $T = V \setminus S$.
  For any $\{\,u,v\,\} \in E$, by definition of an independent set, at most one
  of $u$ and $v$ belongs to $S$, hence, at least one of $u$ and $v$ belongs to
  $T$. Hence, by definition of a vertex cover, $T$ is a vertex cover.

  \paragraph{($\impliedby$)}%
  Let $T$ be a vertex cover in $G$ and define $S = V \setminus T$.
  For any $\{\,u,v\,\} \in E$, by definition of a vertex cover, at least one
  of $u$ and $v$ belongs to $T$, hence, at most one of $u$ and $v$ belongs to
  $S$. Hence, by definition of an independent set, $S$ is an independent set.
\end{proof}

\begin{definition}
  INDEPENDENT-SET $= \{\langle G,k \rangle \st \text{$G$ contains an independent set of size $k$.} \}$
\end{definition}

\begin{definition}
  CLIQUE $= \{\langle G,k \rangle \st \text{$G$ contains a clique of size $k$.} \}$
\end{definition}

\begin{definition}
  VERTEX-COVER $= \{\langle G,k \rangle \st \text{$G$ contains a vertex-cover of size $k$.} \}$
\end{definition}


\begin{theorem}
INDEPENDENT-SET is NP-complete.
\end{theorem}

\begin{proof}
\end{proof}

\section{\texorpdfstring{$\ne$-SAT is NP-complete\footnote{%
Exercise 7.26 from the reference book: Sipser M.,
\emph{Introduction to the Theory of Computation}, 3rd edition
(2013).
In the second edition of the book, the exercise is Exercise 7.24.
}}{$\ne$-SAT is NP-complete}}

\end{document}
