\documentclass{article}
\errorcontextlines 10000

\makeatletter
\usepackage{fontspec}
\usepackage{xunicode}
\usepackage{xltxtra}

\usepackage{fullpage}

\usepackage{hyperref}

\hypersetup{
	linktocpage  = true, %page number is the link (not title)
	colorlinks   = true, %Colours links instead of ugly boxes
	urlcolor     = darkblue, %Colour for external hyperlinks
	linkcolor    = darkblue, %Colour of internal links
	citecolor    = darkblue   %Colour of citations
}

%\hypersetup{
%	linktocpage  = true, %page number is the link (not title)
%	colorlinks   = true, %Colours links instead of ugly boxes
%	allcolors = bookColor,
%	hidelinks = true
%}

% good looking urls
\urlstyle{same}

% we use this for our references as well
\let\sref\ref
\AtBeginDocument{\renewcommand{\ref}[1]{\mbox{\autoref{#1}}}}
\usepackage[nameinlink]{cleveref}

% define colors %
\usepackage[table]{xcolor}

%\usepackage[table,dvipdfx,cmyk]{xcolor}
%\definecolor{bookColor}{cmyk}{0 ,0 ,0 ,1}
%\color{bookColor}

\definecolor{lightgray}{rgb}{.9,.9,.9}
\definecolor{darkgray}{rgb}{.4,.4,.4}
\definecolor{purple}{rgb}{0.65, 0.12, 0.82}
\definecolor{darkblue}{rgb}{0.02, 0.17, 0.40}
\usepackage[stable]{footmisc}
\usepackage{graphicx}
\usepackage{caption}
\usepackage{subcaption}
\usepackage{tikz}
\usepackage{url}
\usepackage{amsmath,amsthm}
\newcommand{\theoremname}{Theorem}
\newtheorem{theorem}{\theoremname}

\newcommand{\st}{\colon\,}
\newcommand{\TM}{TM}
\newcommand{\Atm}{A\textsubscript{TM}}
\newcommand{\HALTtm}{HALT\textsubscript{TM}}

\newcommand*\circled[1]{\tikz[baseline=(char.base)]{
            \node[shape=circle,draw,inner sep=2pt] (char) {#1};}}

% TURING MACHINE DESCRIPTIONS
\usepackage{changepage}
\newenvironment{steps}%
{%
\vspace{0.25cm}%
\begin{adjustwidth}{0.3cm}{0cm}%
\begin{description}%
}
{%
\end{description}%
\end{adjustwidth}%
\vspace{0.1cm}%
}

\newcounter{TMachine}[section]
\renewcommand{\theTMachine}{\thesection.\arabic{TMachine}}%
\newenvironment{TMachine}[1]
  {\refstepcounter{TMachine}%
   \par%
   \vspace{.5\baselineskip\@plus.2\baselineskip\@minus.2\baselineskip}% Space above
   \noindent{#1}%
   \begin{steps}}%\begin{TMachine}
  {\end{steps}%
\vspace{.5\baselineskip\@plus.2\baselineskip\@minus.2\baselineskip}% Space below
}% \end{TMachine}

\newcommand{\accept}{\emph{accept}}
\newcommand{\reject}{\emph{reject}}
\makeatother

\title{Computability and Complexity:\\Exercise Session 3 (2017-10-18)}
\author{Aurélien Ooms\footnote{aureooms@ulb.ac.be}}
\date{\today}

\begin{document}
\maketitle
\tableofcontents

\section{Exercise 4.30\footnote{Exercises from the reference book: Sipser M.,
\emph{Introduction to the Theory of Computation}, 3rd edition (2013). In the
second edition of the book, this exercise is Exercise 4.28.}}

Let \(A\) be a Turing-recognizable language consisting of descriptions of
Turing machines,
\(\{\langle M_1 \rangle, \langle M_2 \rangle, \ldots \}\),
where every \(M_i\) is a decider. Prove that some decidable
language \(Q\) is not decided by any decider \(M_i\) whose description appears in
\(A\).
(Hint: You may find it helpful to consider an enumerator for \(A\).)

\subsection{Solution}

The solution is in two steps. First we construct a new language \(Q\) that is not
decided by any of the \(M_i\) using the technique of diagonalization. Second we
show how to build a decider for \(Q\) using the fact that \(A\) is
Turing-recognizable.

\ref{diagonalization} shows a diagonalization that is similar to the ones we
saw during a previous lecture. We build a new language \(Q\) that disagrees
with each language decided by a decider \(M_i\), that is, \(Q\) disagrees with
\(L(M_i)\) on \(w_i\). Hence, none of the \(M_i\) decides \(Q\).

\begin{table}
\centering
\caption{Building the language \(Q\). Entry \((M,w)\) is \(1\) if \(w \in
L(M)\) and \(0\) otherwise. \(w_i \in Q\) iff \(w_i \not\in L(M_i)\).}
\label{diagonalization}
\begin{tabular}{c | c c c c c c c}
  & \(w_1\) & \(w_2\) & \(w_3\) & \(w_4\) & \(\hdots\) & \(w_i\) & \(\hdots\)\\
  \hline
  \(Q\) & \circled{0} & \circled{1} & \circled{1} & \circled{0} & \(\hdots\) & \circled{0} & \(\hdots\)\\
  \(M_1\) & \circled{1} & 1 & 1 & 1 & \(\hdots\) & 1 & \(\hdots\)\\
  \(M_2\) & 0 & \circled{0} & 0 & 0 & \(\hdots\) & 0 & \(\hdots\)\\
  \(M_3\) & 1 & 0 & \circled{0} & 0 & \(\hdots\) & 0 & \(\hdots\)\\
  \(M_4\) & 0 & 1 & 0 & \circled{1} & \(\hdots\) & 0 & \(\hdots\)\\
  \(\vdots\) & \(\vdots\) & \(\vdots\) & \(\vdots\) & \(\vdots\) & \(\ddots\) & \(\vdots\) & \(\ddots\)\\
  \(M_i\) & 1 & 0 & 1 & 0 & \(\hdots\) & \circled{1} & \(\hdots\)\\
  \(\vdots\) & \(\vdots\) & \(\vdots\) & \(\vdots\) & \(\vdots\) & \(\ddots\) & \(\vdots\) & \(\ddots\)
\end{tabular}
\end{table}

We assume,
without loss of generality, that the order of the \(w_i\) is the standard
string order\footnote{%
This string order is called shortlex or quasi-lexicographic order but is refered to as
``lexicographic order'' in the textbook. See
\url{https://en.wikipedia.org/wiki/Lexicographical\_order} and
\url{https://en.wikipedia.org/wiki/Shortlex_order}.%
}, that is, first ordered by length, then lexicographically.
For a given finite alphabet \(\Sigma\), there are finitely many words of finite
length, hence there is an algorithm \(W\) that enumerates all the words of \(\Sigma^*\)
in the standard string order.

Since \(A\) is Turing-recognizable, there exists an enumerator
for \(A\), as we have shown in a previous lecture.

To prove that \(Q\) is decidable, we build the following \TM{}:

\begin{TMachine}{\(D =\) on input \(w_i\):}
\item[1.] Determine the value of \(i\) using the enumerator \(W\), that is,
enumerate all words in standard string order while maintaining a count of how
many words have been enumerated, and stop when encountering \(w_i\).
\item[2.] Enumerate the \(M_k\) of \(A\) using the enumerator for \(A\). Stop once
we have enumerated \(i\) different deciders, that is, the last decider yielded
by the enumerator is \(M_i\).
\item[3.] Simulate \(M_i\) on \(w_i\). If \(M_i\) accepts, reject. If \(M_i\) rejects, accept.
\end{TMachine}

\(D\) is a decider since:
\begin{enumerate}
\item Steps 1 and 2 execute a finite number of steps since \(i\) is finite.
\item Step 3 stops after a finite number of steps since \(M_i\) is a decider.
\end{enumerate}

\(D\) accepts \(w_i\) iff \(M_i\) rejects \(w_i\). Since \(D\) and \(M_i\) are
deciders, this is equivalent to say
that \(w_i \in L(D)\) iff \(w_i \not\in L(M_i)\), hence \(L(D) = Q\).
\(D\) is thus a decider for \(Q\) and hence \(Q\) is decidable.
\end{document}
