\documentclass{article}
\errorcontextlines 10000

\makeatletter

\usepackage{fontspec}
\usepackage{xunicode}
\usepackage{xltxtra}

\usepackage{fullpage}
\usepackage{mleftright}

\usepackage{hyperref}

\hypersetup{%
	linktocpage  = true, %page number is the link (not title)
	colorlinks   = true, %Colours links instead of ugly boxes
	urlcolor     = darkblue, %Colour for external hyperlinks
	linkcolor    = darkblue, %Colour of internal links
	citecolor    = darkblue   %Colour of citations
}

%\hypersetup{
%	linktocpage  = true, %page number is the link (not title)
%	colorlinks   = true, %Colours links instead of ugly boxes
%	allcolors = bookColor,
%	hidelinks = true
%}

% good looking urls
\urlstyle{same}

\newcommand{\theoremname}{Theorem}
\newcommand{\problemname}{Problem}
\newcommand{\conjecturename}{Conjecture}
\newcommand{\definitionname}{Definition}

% we use this for our references as well
\let\sref\ref
\AtBeginDocument{\renewcommand{\ref}[1]{\mbox{\autoref{#1}}}}
\usepackage[nameinlink]{cleveref}

% define colors %
\usepackage[table]{xcolor}

%\usepackage[table,dvipdfx,cmyk]{xcolor}
%\definecolor{bookColor}{cmyk}{0 ,0 ,0 ,1}
%\color{bookColor}

\definecolor{lightgray}{rgb}{0.9,0.9,0.9}
\definecolor{darkgray}{rgb}{0.4,0.4,0.4}
\definecolor{purple}{rgb}{0.65, 0.12, 0.82}
\definecolor{darkblue}{rgb}{0.02, 0.17, 0.40}
\usepackage[stable]{footmisc}
\usepackage{graphicx}
\usepackage{caption}
\usepackage{subcaption}
\usepackage{tikz}
\usepackage{url}
\usepackage{amsmath,amsthm,amsfonts,mathtools}
\usepackage{xfrac}
\newtheorem{theorem}{\theoremname}
\newtheorem{problem}{\problemname}
\newtheorem{conjecture}{\conjecturename}
\newtheorem{definition}{\definitionname}

% nicely spaced operators
\DeclareMathOperator{\Exists}{\exists}
\DeclareMathOperator{\Forall}{\forall}

\DeclarePairedDelimiter{\ceil}{\lceil}{\rceil}
\newcommand{\card}[1]{|#1|}

\newcommand{\st}{\colon\,}
\newcommand{\TM}{TM}
\newcommand{\Atm}{A\textsubscript{TM}}
\newcommand{\HALTtm}{HALT\textsubscript{TM}}

\newcommand\N{\mathbb{N}_1}

\newcommand*\circled[1]{\tikz[baseline=(char.base)]{
            \node[shape=circle,draw,inner sep=2pt] (char) {#1};}}



% TURING MACHINE DESCRIPTIONS
\usepackage{changepage}
\newenvironment{steps}%
{%
\vspace{0.25cm}%
\begin{adjustwidth}{0.3cm}{0cm}%
\begin{description}%
}
{%
\end{description}%
\end{adjustwidth}%
\vspace{0.1cm}%
}

\newcounter{TMachine}[section]
\renewcommand{\theTMachine}{\thesection.\arabic{TMachine}}%
\newenvironment{TMachine}[1]
  {\refstepcounter{TMachine}%
   \par%
   \vspace{.5\baselineskip\@plus.2\baselineskip\@minus.2\baselineskip}% Space above
   \noindent{#1}%
   \begin{steps}}%\begin{TMachine}
  {\end{steps}%
\vspace{.5\baselineskip\@plus.2\baselineskip\@minus.2\baselineskip}% Space below
}% \end{TMachine}

\newcommand{\accept}{\emph{accept}}
\newcommand{\reject}{\emph{reject}}

\makeatother

\title{Computability and Complexity:\\Exercise Session 5 (2016-10-26)}
\author{Aurélien Ooms\footnote{aureooms@ulb.ac.be}}
\date{\today}

\begin{document}
\maketitle
\tableofcontents

\section{Exercises 7.1 and 7.2\footnote{Exercises from the reference book: Sipser M.,
\emph{Introduction to the Theory of Computation}, 3rd edition (2013).}}
Those exercises are simple to understand once you have understood the definitions for the
\emph{Big-O} and \emph{Small-O} notations\footnote{We saw these definitions
during a previous lecture and it is explained at the beginning of Chapter 7.}.

\subsection{Exercise 7.1}
As a reminder
\begin{definition}
	\(f(n) = O(g(n)) \iff \exists c, n_0\) such that \(f(n) \le c g(n)\) for
	all \(n > n_0\).
\end{definition}

\paragraph{(a)}
\(2n = O(n)\)
is true because \(c = 2\) and \(n_0 = 0\) verify the definition.

\paragraph{(b)}
\(n^2 = O(n)\)
is false because for any \(c \ge 1\) we have that for all \(n > c\), \(n^2 > cn\).

\paragraph{(c)}
\(n^2 = O(n \log^2 n)\)
is false because $n$ grows faster than $\log^2 n$ (hint: check the derivatives).

\paragraph{(d)}
\(n \log n = O(n^2)\)
is true because $\log n$ grows slower than $n$ (hint: check the derivatives).
For instance, \(c = 1\) and \(n_0 = 0\) verify the definition.

\paragraph{(e)}
\(3^n = 2^{O(n)}\)
is true because \(c = \log_2 3\) and \(n_0 = 0\) verify the definition.

\paragraph{(f)}
\(2^{2^n} = O(2^{2^n})\)
is true because \(c = 1\) and \(n_0 = 0\) verify the definition. Note that we
always have \(f(n) = O(f(n))\).

\subsection{Exercise 7.2}
As a reminder
\begin{definition}
	\(f(n) = o(g(n)) \iff \lim_{n \to \infty} \frac{f(n)}{g(n)} = 0\).
\end{definition}

\paragraph{(a)}
\(n = o(2n)\) is false because \(\lim_{n \to \infty} \frac{n}{2n} = \lim_{n \to
\infty} \frac{1}{2} = \frac{1}{2} \neq 0\). Note that we always have \(f(n)
\neq o(k f(n))\) for all constant \(k\).

\paragraph{(b)}
\(2n = o(n^2)\) is true because \(\lim_{n \to \infty} \frac{2n}{n^2} =
\lim_{n \to \infty} \frac{2}{n} = 0\)

\paragraph{(c)}
\(2^n = o(3^n)\) is true because \(\lim_{n \to \infty} \frac{2^n}{3^n} =
\lim_{n \to \infty} \mleft(\frac{2}{3}\mright)^n = 0\).

\paragraph{(d)}
\(1 = o(n)\) is true because \(\lim_{n \to \infty} \frac{1}{n} = 0\).

\paragraph{(e)}
\(n = o(\log n)\) is false because \(\lim_{n \to \infty} \frac{n}{\log n} =
\infty \neq 0\).

\paragraph{(f)}
\(1 = o(\sfrac{1}{n})\) is false because \(\lim_{n \to \infty}
\frac{1}{\sfrac{1}{n}} = \lim_{n \to \infty} n = \infty \neq 0\).

\end{document}
