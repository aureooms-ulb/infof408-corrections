\documentclass{article}
\errorcontextlines 10000

\usepackage{fontspec}
\usepackage{xunicode}
\usepackage{xltxtra}

\usepackage[landscape,twocolumn]{geometry}
\usepackage{fullpage}

\usepackage{hyperref}

\hypersetup{
	linktocpage  = true, %page number is the link (not title)
	colorlinks   = true, %Colours links instead of ugly boxes
	urlcolor     = darkblue, %Colour for external hyperlinks
	linkcolor    = darkblue, %Colour of internal links
	citecolor    = darkblue   %Colour of citations
}

%\hypersetup{
%	linktocpage  = true, %page number is the link (not title)
%	colorlinks   = true, %Colours links instead of ugly boxes
%	allcolors = bookColor,
%	hidelinks = true
%}

% good looking urls
\urlstyle{same}

% we use this for our references as well
\let\sref\ref
\AtBeginDocument{\renewcommand{\ref}[1]{\mbox{\autoref{#1}}}}
\usepackage[nameinlink]{cleveref}

% define colors %
\usepackage[table]{xcolor}

%\usepackage[table,dvipdfx,cmyk]{xcolor}
%\definecolor{bookColor}{cmyk}{0 ,0 ,0 ,1}
%\color{bookColor}

\definecolor{lightgray}{rgb}{.9,.9,.9}
\definecolor{darkgray}{rgb}{.4,.4,.4}
\definecolor{purple}{rgb}{0.65, 0.12, 0.82}
\definecolor{darkblue}{rgb}{0.02, 0.17, 0.40}
\usepackage[stable]{footmisc}
\usepackage{graphicx}
\usepackage{caption}
\usepackage{subcaption}
\usepackage{tikz}
\usepackage{url}
\usepackage{amsmath}

\title{Scooping the Loop Snooper}
\author{Geoffrey K. Pullum}
\date{}
\begin{document}
\maketitle

\begin{verse}
\emph{No general procedure for bug checks will do.}\\
Now, I won't just assert that, I’ll prove it to you.\\
I will prove that although you might work till you drop,\\
you cannot tell if computation will stop.
\end{verse}
\begin{verse}
For imagine we have a procedure called \(P\)\\
that for specified input permits you to see\\
whether specified source code, with all of its faults,\\
defines a routine that eventually halts.
\end{verse}
\begin{verse}
You feed in your program, with suitable data,\\
and \(P\) gets to work, and a little while later\\
(in finite compute time) correctly infers\\
whether infinite looping behavior occurs.
\end{verse}
\begin{verse}
If there will be no looping, then \(P\) prints out ``Good.''\\
That means work on this input will halt, as it should.\\
But if it detects an unstoppable loop,\\
then \(P\) reports ``Bad!'' -- which means you're in the soup.
\end{verse}
\begin{verse}
Well, the truth is that \(P\) cannot possibly be,\\
because if you wrote it and gave it to me,\\
I could use it to set up a logical bind\\
that would shatter your reason and scramble your mind.
\end{verse}
\begin{verse}
Here's the trick that I’ll use -- and it's simple to do.\\
I’ll define a procedure, which I will call \(Q\),\\
that will use \(P\)'s predictions of halting success\\
to stir up a terrible logical mess.
\end{verse}
\begin{verse}
For a specified program, say \(A\), one supplies,\\
the first step of this program called \(Q\) I devise\\
is to find out from \(P\) what's the right thing to say\\
of the looping behavior of \(A\) run on \(A\).
\end{verse}
\begin{verse}
If \(P\)'s answer is ``Bad!'', \(Q\) will suddenly stop.\\
But otherwise, \(Q\) will go back to the top,\\
and start off again, looping endlessly back,\\
till the universe dies and turns frozen and black.
\end{verse}
\begin{verse}
And this program called \(Q\) wouldn't stay on the shelf;\\
I would ask it to forecast its run on itself.\\
When it reads its own source code, just what will it do?\\
What's the looping behavior of \(Q\) run on \(Q\)?
\end{verse}
\begin{verse}
If \(P\) warns of infinite loops, \(Q\) will quit;\\
yet \(P\) is supposed to speak truly of it!\\
And if \(Q\)'s going to quit, then \(P\) should say ``Good.''\\
Which makes \(Q\) start to loop! (\(P\) denied that it would.)
\end{verse}
\begin{verse}
No matter how \(P\) might perform, \(Q\) will scoop it:\\
\(Q\) uses \(P\)'s output to make \(P\) look stupid.\\
	Whatever \(P\) says, it cannot predict \(Q\):\\
\(P\) is right when it's wrong, and is false when it's true!
\end{verse}
\begin{verse}
I’ve created a paradox, neat as can be --\\
and simply by using your putative \(P\).\\
When you posited \(P\) you stepped into a snare;\\
Your assumption has led you right into my lair.
\end{verse}
\begin{verse}
So where can this argument possibly go?\\
I don't have to tell you; I'm sure you must know.\\
A \textbf{reductio}: There cannot possibly be\\
a procedure that acts like the mythical \(P\).
\end{verse}
\begin{verse}
You can never find general mechanical means\\
for predicting the acts of computing machines;\\
it's something that cannot be done. So we users\\
must find our own bugs. Our computers are losers!
\end{verse}
\end{document}
