\section{CLIQUE, VERTEX-COVER, and INDEPENDENT-SET}

Let us first define the concepts of clique and independent set in a graph.
Given a graph, a clique is a subset of its vertices such that \textbf{each} pair in this
subset is connected by an edge.

\begin{definition}[Clique]
  Let $G=(V,E)$, $S\subseteq V$. The set of vertices $S$ is a clique of $G$
  if and only if for all $(u,v) \in S^2$ with $u\ne v$, $\{\,u,v\,\} \in E$.
\end{definition}

Given a graph, an independent set is a subset of its vertices such that
\textbf{no} pair in this
subset is connected by an edge.
\begin{definition}[Independent set]
  Let $G=(V,E)$, $S\subseteq V$. The set of vertices $S$ is an independent set
  of $G$ if and only if for all $(u,v) \in S^2$ with $u\ne v$, $\{\,u,v\,\} \not\in E$.
\end{definition}

We show that, as you may already have noticed, those two concepts are
closely related. To achieve this goal, let us define the complement of a graph
$G$, $\bar{G}$, that has the same vertex set as $G$, but whose edge set
contains only the edges that are missing in $G$:
\begin{definition}[Complement of a graph]
  Let $G = (V,E)$, define $\bar{E}=\{\,\{\,u,v\,\} \st (u,v) \in S^2,
  u \ne v, \{\,u,v\,\} \not\in E\,\}$ and $\bar{G} = (V,\bar{E})$.
\end{definition}

We are now ready to prove our first result:
\begin{theorem}\label{is-cl}
  Let $G = (V,E)$, $S \subseteq V$. The set $S$ is an independent set in $G$ if
  and only if $S$ is a clique in $\bar{G}$.
\end{theorem}

\begin{proof}
  Let us prove the two directions of the implication.
  \paragraph{($\implies$)}%
  Let $S$ be an independent set in $G$.
  For any $(u,v) \in S^2, u \ne v$, by definition of an independent set,
  $\{\,u,v\,\} \not\in E$, hence, $\{\,u,v\,\} \in \bar{E}$. Hence, by
  definition of a clique, $S$ is a clique in $\bar{G}$.

  \paragraph{($\impliedby$)}%
  Conversely, let $S$ be a clique in $\bar{G}$.
  For any $(u,v) \in S^2, u \ne v$, by definition of a clique,
  $\{\,u,v\,\} \in \bar{E}$, hence, $\{\,u,v\,\} \not\in E$. Hence, by
  definition of an independent set, $S$ is an independent set in $G$.
\end{proof}

We define a third concept, the vertex cover. Given a graph, a vertex cover in
this graph covers all its edges by containing at least one endpoint of each.
\begin{definition}[Vertex cover]
  Let $G=(V,E)$, $S\subseteq V$. The set of vertices $S$ is a vertex cover of
  $G$ if and only if for all $\{u,v\} \in E$, $u \in S \lor v \in S$.
\end{definition}

Again, although maybe less obvious, the concepts of vertex cover and
independent set are closely related.
\begin{theorem}\label{is-vc}
  Let $G = (V,E)$, $S \subseteq V$. The set $S$ is an independent set in $G$ if
  and only if $T = V \setminus S$ is a vertex cover in $G$.
\end{theorem}

\begin{proof}
  Let us prove the two directions of the implication.
  \paragraph{($\implies$)}%
  Let $S$ be an independent set in $G$ and define $T = V \setminus S$.
  For any $\{\,u,v\,\} \in E$, by definition of an independent set, at most one
  of $u$ and $v$ belongs to $S$, hence, at least one of $u$ and $v$ belongs to
  $T$. Hence, by definition of a vertex cover, $T$ is a vertex cover in $G$.

  \paragraph{($\impliedby$)}%
  Conversely, let $T$ be a vertex cover in $G$ and define $S = V \setminus T$.
  For any $\{\,u,v\,\} \in E$, by definition of a vertex cover, at least one
  of $u$ and $v$ belongs to $T$, hence, at most one of $u$ and $v$ belongs to
  $S$. Hence, by definition of an independent set, $S$ is an independent set in
  $G$.
\end{proof}

As an exercise, you can show the following:
\begin{corollary}\label{vc-cl}
  Let $G = (V,E)$, $S \subseteq V$. The set $S$ is an clique in $G$ if
  and only if $T = V \setminus S$ is a vertex cover in $\bar{G}$.
\end{corollary}

We are interested in the complexities of the problems of deciding the existence
of those substructures in arbitrary graphs. We define the following languages:
\begin{definition}
  CLIQUE $= \{\langle G,k \rangle \st \text{$G$ is a graph and contains a clique of size $k$.} \}$
\end{definition}

\begin{definition}
  INDEPENDENT-SET $= \{\langle G,k \rangle \st \text{$G$ is a graph and contains an independent set of size $k$.} \}$
\end{definition}

\begin{definition}
  VERTEX-COVER $= \{\langle G,k \rangle \st \text{$G$ is a graph and contains a vertex-cover of size $k$.} \}$
\end{definition}

During a previous lecture, you proved that CLIQUE is complete for the
complexity class NP. This means that (1) CLIQUE is in NP, that is, it is as
``easy'' as all other problems in NP, and (2), CLIQUE is NP-hard,
that is, it is as ``hard'' as all other problems in NP.
\begin{theorem}\label{cl:npc}
CLIQUE is NP-complete.
\end{theorem}
\begin{proof}
  See lecture notes.
\end{proof}

Building on~\ref{is-cl}~and~\ref{cl:npc}, we can prove the following:
\begin{theorem}\label{is:npc}
INDEPENDENT-SET is NP-complete.
\end{theorem}

\begin{proof}
  We must prove that INDEPENDENT-SET is both in NP and NP-complete.
  \paragraph{$\in$ NP}
  By~\ref{is-cl}, $\langle G,k \rangle \in$ INDEPENDENT-SET if (and only if)
  $\langle \bar{G},k\rangle \in$ CLIQUE.
  In other words, deciding if a graph $G$ contains an independent set of size $k$ is
  equivalent to deciding if the complement of $G$, $\bar{G}$, contains a
  clique of size $k$. The reduction takes quadratic time in the worst
  case where $G$ contains no edge and $\bar{G}$ contains $\binom{\card{V}^2}{2}$.
  By~\ref{cl:npc}, CLIQUE is in NP, that is, ``easy''. Since
  we can solve an instance of INDEPENDENT-SET by solving a polynomial size instance of
  CLIQUE then INDEPENDENT-SET must be as ``easy'' as CLIQUE, that
  is, INDEPENDENT-SET is in NP.

  \paragraph{NP-hard}
  By~\ref{is-cl}, $\langle G,k \rangle \in$ CLIQUE if (and only if)
  $\langle \bar{G},k\rangle \in$ INDEPENDENT-SET.
  In other words, deciding if a graph $G$ contains a clique of size $k$ is
  equivalent to deciding if the complement of $G$, $\bar{G}$, contains an
  independent set of size $k$. The reduction takes quadratic time in the worst
  case where $G$ contains no edge and $\bar{G}$ contains $\binom{\card{V}^2}{2}$.
  By~\ref{cl:npc}, CLIQUE is NP-hard, that is, ``hard''. Since
  we can solve an instance of CLIQUE by solving a polynomial size instance of
  INDEPENDENT-SET then INDEPENDENT-SET must be as ``hard'' as CLIQUE, that
  is, INDEPENDENT-SET is NP-hard.
\end{proof}

Building on~\ref{is-vc}~and~\ref{is:npc}, we can prove the following:
\begin{theorem}\label{vc:npc}
VERTEX-COVER is NP-complete.
\end{theorem}

\begin{proof}
  We must prove that VERTEX-COVER is both in NP and NP-complete.
  \paragraph{$\in$ NP}
  By~\ref{is-vc}, $\langle G,k \rangle \in$ VERTEX-COVER if (and only if)
  $\langle G,\card{V}-k\rangle \in$ INDEPENDENT-SET.
  In other words, deciding if a graph $G$ contains a vertex cover of size $k$ is
  equivalent to deciding if $G$ contains an
  independent set of size $\card{V}-k$. The reduction takes linear time as $G$ stays the
  same and $\card{V}-k$ has bit-size $\lceil\log \card{V} - k\rceil \le
  \max(\log\card{V},\log k) + 1$.
  By~\ref{is:npc}, INDEPENDENT-SET is in NP, that is, ``easy''. Since
  we can solve an instance of VERTEX-COVER by solving a polynomial size instance of
  INDEPENDENT-SET then VERTEX-COVER must be as ``easy'' as INDEPENDENT-SET, that
  is, VERTEX-COVER is in NP.

  \paragraph{NP-hard}
  By~\ref{is-vc}, $\langle G,k \rangle \in$ INDEPENDENT-SET if (and only if)
  $\langle G,\card{V}-k\rangle \in$ VERTEX-COVER.
  In other words, deciding if a graph $G$ contains an independent set of size $k$ is
  equivalent to deciding if $G$ contains a
  vertex cover of size $\card{V}-k$. The reduction takes linear time as $G$ stays the
  same and $\card{V}-k$ has bit-size $\lceil\log \card{V} - k\rceil \le
  \max(\log\card{V},\log k) + 1$.
  By~\ref{is:npc}, INDEPENDENT-SET is NP-hard, that is, ``hard''. Since
  we can solve an instance of INDEPENDENT-SET by solving a polynomial size instance of
  VERTEX-COVER then VERTEX-COVER must be as ``hard'' as INDEPENDENT-SET, that
  is, VERTEX-COVER is NP-hard.
\end{proof}
