\section{Exercise 4.17\footnote{Exercise 4.16 in the second edition of the
reference book}}
A DFA is a \(5\)-tuple \((Q,\Sigma,\delta,q_0,F)\) where \(Q\) is the set of
states, \(\Sigma\) is the input alphabet, \(\delta : Q \times \Sigma \to Q\) is the transition
function, \(q_0 \in Q\) is the start state and \(F \subseteq Q\) is the set of
accept states. Without loss of generality, we consider the case \(\Sigma =
\{0,1\}\).
Given a DFA \(A\) and an input string \(w = w_1 w_2 \ldots w_n\) we can check
whether \(w \in L(A)\) by computing the value of the following function \(D_A\)
for \(w\)
\begin{align*}
	D_A(w) &= \alpha_A(D_A'(w))\\
	\alpha_A(q) &= \begin{cases}
		\accept & \text{if \(q \in F\)},\\
		\reject & \text{otherwise}
	\end{cases}\\
	D_A'(w) &= D_A''(q_0,w)\\
	D_A''(q,w_1 w_{2 \ldots n}) &= D_A''(\delta(q,w_1),w_{2 \ldots n})\\
	D_A''(q,\epsilon) &= q.
\end{align*}

Given two DFA's \(A_1 = (Q_1,\Sigma,\delta_1,q_0^1,F_1)\) and \(A_2 =
(Q_2,\Sigma,\delta_2,q_0^2,F_2)\) we want to check whether \(L(A_1) =
L(A_2)\). Imagine we are given an input \(w\) and run \(A_1\) and \(A_2\)
on their own copy of \(w\). Let \(e_1 = D_{A_1}'(w)\) be the last state reached by
\(A_1\) when run on \(w\) and let \(e_2 = D_{A_2}'(w)\) be the last state reached
by \(A_2\) when run on \(w\). Then \(L(A_1) = L(A_2) \implies D_{A_1}(w) =
D_{A_2}(w)\), that is, either both \(e_1\) and \(e_2\) are accepting, or
both are rejecting. Indeed, to show that \(L(A_1) = L(A_2)\) we only need to
show that each possible ending configuration \((e_1,e_2) \in (Q_1 \times Q_2)\) obtained by
simulating both \(A_1\) and \(A_2\) on an input word contain states that are
both accepting or rejecting. There are at most \(\card{Q_1} \card{Q_2}\) such
configurations. Here is an algorithm that checks all of them
\begin{TMachine}{On input \(\langle A_1, A_2 \rangle\):}
\item[1.] \(Z \gets \{\}\)
\item[2.] \(X \gets \{(q_0^1,q_0^2)\}\)
\item[3.] While \(X \neq \emptyset\):
\item[3.1.] Pick \((q_1,q_2) \in X\).
\item[3.2.] If \(\alpha_{A_1}(q_1) \neq \alpha_{A_2}(q_2)\), \reject.
\item[3.3.] \(Z \gets Z \cup \{(q_1,q_2)\}\)
\item[3.4.] \(X \gets X \cup
\{(\delta_1(q_1,0),\delta_2(q_2,0)),(\delta_1(q_1,1),\delta_2(q_2,1))\}
\setminus Z\)
\item[4.] \accept.
\end{TMachine}
Another equivalent (but less efficient) algorithm checks that \(A_1\) agrees
with \(A_2\) on all strings of length at most \(\card{Q_1} \card{Q_2}\).
% using the pigeon hole principle to prove this would be a better idea
