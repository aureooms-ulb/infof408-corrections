\section{\texorpdfstring{TQBF is in PSPACE\footnote{%
First part in Theorem 8.9 from the reference book: Sipser M.,
\emph{Introduction to the Theory of Computation}, 3rd edition
(2013).}}{TQBF is in PSPACE}}

As a reminder
\begin{definition}
	SPACE\((f(n))\) \(= \{L \st L \) is a language decided by an \(O(f(n))\) space
	deterministic Turing machine.\(\}\)
\end{definition}
\begin{definition}
	PSPACE \(= \bigcup_k\) SPACE\((n^k)\)
\end{definition}
\begin{definition}
	TQBF \(= \{ \langle \phi \rangle \st \phi\) is a true fully quantified
	boolean formula.\(\}\)
\end{definition}

\begin{theorem}
	TQBF \(\in\) PSPACE
\end{theorem}

\begin{proof}
Obviously, the following recursive algorithm decides TQBF
\begin{TMachine}{\(T =\) ``On input \(\langle \varphi \rangle\), a fully quantified Boolean formula:}
\item[1.] If \(\varphi\) contains no quantifiers, then it is an expression with only
constants, so evaluate \(\varphi\) and accept if it is true; otherwise, reject.
\item[2.] If \(\varphi\) equals \(\exists x~\phi\), recursively call \(T\) on
	\(\phi\), first with \(0\) substituted
	for \(x\) and then with \(1\) substituted for \(x\). If either result is accept,
then \accept; otherwise, \reject.
\item[3.] If \(\varphi\) equals \(\forall x~\phi\), recursively call \(T\) on
	\(\phi\), first with \(0\) substituted
	for \(x\) and then with \(1\) substituted for \(x\). If both results are
	accept, then \accept; otherwise, \reject.''
\end{TMachine}

The depth of the recursion is \(m\), the number of variables.
Note that \(m = O(n)\), where \(n\) is the size of the input formula.
At each level of
the recursion we only need to store the value of one variable and so the space
used is \(O(n)\). Hence, this algorithm decides TQBF using linear space.

\end{proof}
