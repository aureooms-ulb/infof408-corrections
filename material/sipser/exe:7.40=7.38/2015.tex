\section{\texorpdfstring{Exercise 7.40\footnote{In the
second edition of the book, this exercise is Exercise 7.38.}}{Exercise 7.40}}

Let \(D\) be a decider for CLIQUE, here is an algorithm that finds the maximum
clique of an input graph \(G\) with vertices \(v_1,v_2,\ldots,v_n\)
\begin{TMachine}{\(M =\) on input \(\langle G \rangle\)}
\item[1.] For \(k=2\) to \(n\):
\item[1.1.] Decide whether \(G\) has a \(k\)-clique using \(D\).
\item[1.2.] If \(D\) rejects, let \(k \gets k - 1\) and go to step \textbf{2}.
\item[2.] For \(i=1\) to \(n\):
\item[2.1.] Decide whether \(G \setminus \{x_i\}\) has a \(k\)-clique using \(D\).
\item[2.2.] If \(D\) accepts, let \(G \gets G \setminus \{x_i\}\).
\item[3.] Output \(G\).
\end{TMachine}
Suppose P = NP. Then there exists a decider \(D\) for CLIQUE that runs in
polynomial time. Then the algorithm given above runs in polynomial time.
