\section{Exercise 3.15.a}
A decision problem can be seen as language \(A\) using the following mapping:
\begin{enumerate}
	\item To each instance of the problem is associated a word of some alphabet
		\(\Sigma\) that encodes this instance.
	\item \(A\) is the language (a set of words) containing only
		the words that encode an instance for
		which the decision problem answer is ``yes''. Words encoding an
		instance for which the decision problem is ``no'' or words that do not
		encode an instance (garbage) are excluded from the language.
\end{enumerate}
Deciding the answer to an instance of the problem amounts to determining
whether the word encoding this instance belongs to \(A\).

A \TM{} decides a language if for each word \(w \in \Sigma^{*}\)
of the language alphabet \(\Sigma\)
it either accepts or rejects the word after a finite number of steps. A decider
will never loop forever.

\(A \cup B = \{w \st w \in A \lor w \in B\}\)
is\footnote{\(\lor = \text{``or''}\)}
the language that contains all words in \(A\) and all words in \(B\).

If we are given a \TM{} \(M_A\) that decides a language \(A\) and
a \TM{} \(M_B\) that decides a language \(B\) we can decide the
language \(A \cup B\) using \(M_{A \cup B}\) defined as follows
(solution given in the textbook):


\begin{TMachine}{``On input word \(w\):}
\item[1.] Run \(M_A\) on \(w\). If it accepts, \accept.
\item[2.] Run \(M_B\) on \(w\). If it accepts, \accept. Otherwise, \reject.''
\end{TMachine}
\(M_{A \cup B}\) accepts \(w\) if either \(M_A\) or \(M_B\) accepts it.
If both reject, \(M_{A \cup B}\) rejects.

\section{Exercise 3.16.a}

The difference from the previous exercise is that we are given \(M_A\) and
\(M_B\) that are recognizers (not deciders). A recognizer \(M\) will always stop
after a finite number of steps if run on a word that belongs to the language
\(L(M)\) (the language recognized by \(M\)).
However, if the word does not belong to \(L(M)\), \(M\) can either stop and reject or loop forever.

The solution is thus to simulate both machines \(M_A\) and \(M_B\) ``in
parallel'' (instead of in ``sequence'', like above). Solution given in the textbook:

\begin{TMachine}{``On input w:}
\item[1.] Run \(M_A\) and \(M_B\) alternately on \(w\) step by step. If either accepts,
	\accept{}. If both halt and reject, \reject{}.''
\end{TMachine}

If either \(M_A\) or \(M_B\) accepts \(w\), \(M_{A \cup B}\) accepts \(w\)
because the accepting \TM{} arrives to its
accepting state after a finite number of steps. Note that if both \(M_A\) and
\(M_B\) reject and either of them does so by looping, then \(M_{A \cup B}\)
will loop.

\section{Other exercises from 3.15 and 3.16}

You can try to solve these exercises by yourself. Here are the definitions of
the operators\footnote{\(\land = \text{``and''}\)}:
\begin{itemize}
\item[concatenation] \( A \circ B = \{ w_a w_b \st w_a \in A \land w_b \in
	B\}\), that is, the
	words that are the result of concatenating a word of \(B\) to a word of
	\(A\).
\item[star] \( A^* = \{ \epsilon \} \cup \{ w_1 \cdots w_n \st w_i \in A, n \ge
	1\}\), that is,
	words that are the result of concatenating any number of words from \(A\).
\item[complementation] \( \overline{A} = \Sigma^{*} \setminus A = \{ w \in \Sigma^{*} \st w \not \in A
	\}\), that is, words containing only symbols of the language alphabet that are not in the
	language.
\item[intersection] \( A \cap B = \{w \st w \in A \land w \in B\}\), that
	is, words that are in both \(A\) and \(B\).
\item[homomorphism] Let \(h \colon \Sigma^{*} \to \Sigma^{*}\)
	be a function such that \(h(w_1 w_2) = h(w_1) h(w_2)\) for any words
	\(w_1\) and \(w_2\). If \(A\) is
	a language, then \(h(A)\) is defined as \(\{ w \st w = h(x), x \in A\}\).
\end{itemize}

It is important to understand why complementation is not listed in Exercise 3.16.
