\section{Exercise 5.9}

We showed during last lecture (2016-10-10) that given some non-trivial property
\(P\) the problem of deciding whether the language \(L(M)\) of some TM \(M\)
has property \(P\), that is, \(L(M) \in P\), is undecidable. This result is
Rice's theorem. To solve this exercise, we just have to show
that\footnote{\(w = w_1 w_2 \ldots w_n \iff w^{\mathcal{R}} = w_{n} w_{n-1}\ldots w_{1}\)}
\(P = \{A \st w \in A \iff
w^{\mathcal{R}} \in A \}\) is indeed a non-trivial\footnote{Non-trivial means
that \(P \neq \emptyset \land P \neq \mathcal{P}(\Sigma^*)\), where
\(\mathcal{P}(S)\) is the set of all subsets of \(S\). \(\mathcal{P}\) is
called the powerset operator.} property. For
example, \(\{\texttt{01}\} \not\in P\) so \(P \neq \mathcal{P}(\Sigma^*)\) and
\(\{\texttt{1}\} \in P\) so \(P \neq \emptyset\).

Alternatively, we can produce the following proof reducing \Atm{} to our problem.
Let us assume \(T = \{\langle M \rangle \st M \text{ is a \TM{} that accepts
\(w^{\mathcal{R}}\) whenever it accepts \(w\)}\}\) is decidable. That means
there exists a decider \(H\) for language \(T\). Given some input \(\langle
M,w\rangle\) to test against the language \Atm{}, build the following
\TM{}:

\begin{TMachine}{\(D_{\langle M,w\rangle} =\) on input \(x\):}
\item[1.] If \(x = \texttt{01}\): \accept.
\item[2.] Otherwise run \(M\) on \(w\). If \(M\) accepts, \accept. Otherwise
		\reject.
\end{TMachine}
Note that \(L(D) = \Sigma^* \in P\) if and only if \(M\) accepts \(w\) and
\(L(D) = \{\texttt{01}\} \not\in P\) if and only if \(M\) does not accept
\(w\). Hence, for any input \(\langle M,w\rangle\) we can decide whether \(M\)
accepts \(w\) by running the decider \(H\) on \(\langle D_{\langle M,w\rangle}
\rangle\), a contradiction with respect to the theorem about the undecidability
of \Atm{} we proved during the lectures.
