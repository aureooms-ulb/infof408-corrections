\section{\texorpdfstring{$\ne$SAT\footnote{%
Exercise 7.26 from the reference book: Sipser M.,
\emph{Introduction to the Theory of Computation}, 3rd edition
(2013).
In the second edition of the book, the exercise is Exercise 7.24.
}}{$\ne$SAT}}

Let us start with a definition:
\begin{definition}
  Let $\phi$ be a 3cnf-formula. An $\ne$-assignment to the variables of $\phi$
  is one where each clause contains two literals with unequal truth values.
\end{definition}
In other words, an $\ne$-assignment satisfies $\phi$ without assigning three
true literals in any clause.

\begin{theorem}\label{neg}
  The negation of any $\ne$-assignment to $\phi$ is also an $\ne$-assignment.
\end{theorem}
\begin{proof}
  An $\ne$-assignment to the variables of $\phi$ is such that each clause
  contains a true and a false literal. Hence, the negation of that
  $\ne$-assignement is such that each clause of $\phi$ contains a false
  and a true literal, that is, it is an $\ne$-assignment.
\end{proof}

Let $\ne$SAT be the collection of 3cnf-formulas that have an $\ne$-assignment.
\begin{definition}
  $\ne$SAT = $\{\,\langle \phi \rangle\st \text{$\phi$ has an $\ne$-assignment.} \,\}$
\end{definition}

We show there is a polynomial time reduction from 3SAT to $\ne$SAT.
\begin{theorem}
  3SAT $\le_P$ $\ne$SAT.
\end{theorem}

\begin{proof}
  Given $\varphi$ a 3cnf-formula, let $\phi$ be the formula where we replace
  each clause $c_i = (x \lor y \lor z)$ of $\varphi$, with the two clauses $(x
  \lor y \lor \alpha_i)$ and $(\bar{\alpha_i} \lor z \lor \beta)$, where $\alpha_i$ is a
  new variable for each clause $c_i$, and $\beta$ is a single additional new
  variable common to every second clause in $\phi$.

  Let us check that $\varphi$ is satisfiable if and only if $\phi$ has an
  $\ne$-assignment.

  \paragraph{($\implies$)} If $\varphi$ is satisfiable, set $\beta=F$, for each
  $c_i$, looking at~\ref{table}, if $x=F$, $y=F$, $z=T$, set $\alpha_i=T$,
  otherwise set $\alpha_i=F$.

  \paragraph{($\impliedby$)}
  If $\phi$ has an $\ne$-assignment, then, by~\ref{neg}, it has an
  $\ne$-assignment with $\beta$ set to false. Looking at the $\ne$-assignments
  with $\beta$ set to false in~\ref{table}, we see that in all cases $\varphi$
  is satisfiable.

\begin{table}
	\centering
	\caption{Truth table for a clause in $\varphi$ and its corresponding clauses in
	$\phi$.}\label{table}
	\begin{tabular}{|c|ccc|c|cc|c|}
	\hline
	& $x$ & $y$ & $z$ & $(x \lor y \lor z)$ & $\alpha_i$ & $\beta$ & $(x \lor
	y \lor \alpha_i) \land (\bar{\alpha_i} \lor z \lor \beta)$ \\
	\hline
	1 & F & F & F & F & F & F & F \\
	2 & F & F & T & T & F & F & F \\
	3 & F & T & F & T & F & F & T \\
	4 & F & T & T & T & F & F & T \\
	5 & T & F & F & T & F & F & T \\
	6 & T & F & T & T & F & F & T \\
	7 & T & T & F & T & F & F & T \\
	8 & T & T & T & T & F & F & T \\
	\hline
	9 & F & F & F & F & T & F & F \\
	10 & F & F & T & T & T & F & T \\
	11 & F & T & F & T & T & F & F \\
	12 & F & T & T & T & T & F & T \\
	13 & T & F & F & T & T & F & F \\
	14 & T & F & T & T & T & F & T \\
	15 & T & T & F & T & T & F & F \\
	16 & T & T & T & T & T & F & F \\
	\hline
	%17 & F & F & F & F & F & T & F \\
	%18 & F & F & T & T & F & T & F \\
	%19 & F & T & F & T & F & T & T \\
	%20 & F & T & T & T & F & T & F \\
	%21 & T & F & F & T & F & T & T \\
	%22 & T & F & T & T & F & T & F \\
	%23 & T & T & F & T & F & T & T \\
	%24 & T & T & T & T & F & T & F \\
	%\hline
	%25 & F & F & F & F & T & T & T \\
	%26 & F & F & T & T & T & T & T \\
	%27 & F & T & F & T & T & T & T \\
	%28 & F & T & T & T & T & T & T \\
	%29 & T & F & F & T & T & T & T \\
	%30 & T & F & T & T & T & T & T \\
	%31 & T & T & F & T & T & T & T \\
	%32 & T & T & T & T & T & T & F \\
	%\hline
\end{tabular}
\end{table}

The reduction takes linear time: $\phi$ is exactly two times longer than $\varphi$.
\end{proof}

Given a 3cnf-formula $\phi$, it is easy to see that an $\ne$-assignment to
$\phi$ is a polynomial size certificate for the belonging of $\phi$ to $\ne$SAT
that can be verified in polynomial time. Hence,
\begin{theorem}
  $\ne$SAT is NP-complete.
\end{theorem}
