\section{Exercise 4.30\footnote{Exercises from the reference book: Sipser M.,
\emph{Introduction to the Theory of Computation}, 3rd edition (2013). In the
second edition of the book, this exercise is Exercise 4.28.}}

Let \(A\) be a Turing-recognizable language consisting of descriptions of
Turing machines,
\(\{\langle M_1 \rangle, \langle M_2 \rangle, \ldots \}\),
where every \(M_i\) is a decider. Prove that some decidable
language \(Q\) is not decided by any decider \(M_i\) whose description appears in
\(A\).
(Hint: You may find it helpful to consider an enumerator for \(A\).)

\subsection{Solution}

The solution is in two steps. First we construct a new language \(Q\) that is not
decided by any of the \(M_i\) using the technique of diagonalization. Second we
show how to build a decider for \(Q\) using the fact that \(A\) is
Turing-recognizable.

\ref{diagonalization} shows a diagonalization that is similar to the ones we
saw during a previous lecture. We build a new language \(Q\) that disagrees
with each language decided by a decider \(M_i\), that is, \(Q\) disagrees with
\(L(M_i)\) on \(w_i\). Hence, none of the \(M_i\) decides \(Q\).

\begin{table}
\centering
\caption{Building the language \(Q\). Entry \((M,w)\) is \(1\) if \(w \in
L(M)\) and \(0\) otherwise. \(w_i \in Q\) iff \(w_i \not\in L(M_i)\).}
\label{diagonalization}
\begin{tabular}{c | c c c c c c c}
  & \(w_1\) & \(w_2\) & \(w_3\) & \(w_4\) & \(\hdots\) & \(w_i\) & \(\hdots\)\\
  \hline
  \(Q\) & \circled{0} & \circled{1} & \circled{1} & \circled{0} & \(\hdots\) & \circled{0} & \(\hdots\)\\
  \(M_1\) & \circled{1} & 1 & 1 & 1 & \(\hdots\) & 1 & \(\hdots\)\\
  \(M_2\) & 0 & \circled{0} & 0 & 0 & \(\hdots\) & 0 & \(\hdots\)\\
  \(M_3\) & 1 & 0 & \circled{0} & 0 & \(\hdots\) & 0 & \(\hdots\)\\
  \(M_4\) & 0 & 1 & 0 & \circled{1} & \(\hdots\) & 0 & \(\hdots\)\\
  \(\vdots\) & \(\vdots\) & \(\vdots\) & \(\vdots\) & \(\vdots\) & \(\ddots\) & \(\vdots\) & \(\ddots\)\\
  \(M_i\) & 1 & 0 & 1 & 0 & \(\hdots\) & \circled{1} & \(\hdots\)\\
  \(\vdots\) & \(\vdots\) & \(\vdots\) & \(\vdots\) & \(\vdots\) & \(\ddots\) & \(\vdots\) & \(\ddots\)
\end{tabular}
\end{table}

We assume,
without loss of generality, that the order of the \(w_i\) is the standard
string order\footnote{%
This string order is called shortlex or quasi-lexicographic order but is refered to as
``lexicographic order'' in the textbook. See
\url{https://en.wikipedia.org/wiki/Lexicographical\_order} and
\url{https://en.wikipedia.org/wiki/Shortlex_order}.%
}, that is, first ordered by length, then lexicographically.
For a given finite alphabet \(\Sigma\), there are finitely many words of finite
length, hence there is an algorithm \(W\) that enumerates all the words of \(\Sigma^*\)
in the standard string order.

Since \(A\) is Turing-recognizable, there exists an enumerator
for \(A\), as we have shown in a previous lecture.

To prove that \(Q\) is decidable, we build the following \TM{}:

\begin{TMachine}{\(D =\) on input \(w_i\):}
\item[1.] Determine the value of \(i\) using the enumerator \(W\), that is,
enumerate all words in standard string order while maintaining a count of how
many words have been enumerated, and stop when encountering \(w_i\).
\item[2.] Enumerate the \(M_k\) of \(A\) using the enumerator for \(A\). Stop once
we have enumerated \(i\) different deciders, that is, the last decider yielded
by the enumerator is \(M_i\).
\item[3.] Simulate \(M_i\) on \(w_i\). If \(M_i\) accepts, reject. If \(M_i\) rejects, accept.
\end{TMachine}

\(D\) is a decider since:
\begin{enumerate}
\item Steps 1 and 2 execute a finite number of steps since \(i\) is finite.
\item Step 3 stops after a finite number of steps since \(M_i\) is a decider.
\end{enumerate}

\(D\) accepts \(w_i\) iff \(M_i\) rejects \(w_i\). Since \(D\) and \(M_i\) are
deciders, this is equivalent to say
that \(w_i \in L(D)\) iff \(w_i \not\in L(M_i)\), hence \(L(D) = Q\).
\(D\) is thus a decider for \(Q\) and hence \(Q\) is decidable.
