\section{3SAT is polynomial-time reducible to CLIQUE\footnote{%
Theorem 7.32 from the reference book: Sipser M.,
\emph{Introduction to the Theory of Computation}, 3rd edition (2013).}}

\subsection{Introduction}

Let us recall the big picture. We have shown in the first part of the course,
the part about computability, that not all problems can be solved by means of
Turing machines. We made the distinction between decidable and undecidable
problems, the former being the ones we can solve with algorithms, that is,
procedures that eventually terminate.

The second part, that is, the part about complexity, focuses on the resources
it takes to solve decidable problems. We already introduced the classes P and
NP\@. NP is the class of problems for which there exists a nondeterministic
polynomial-time algorithm. An equivalent definition is to consider the
languages for which all words have at least one corresponding certificate of
appartenance of polynomial size that can be verified using a polynomial-time
algorithm called a verifier. P is the class of problems for which there exists
a deterministic polynomial-time algorithm. By definition P is contained in NP\@.
However, we do not know whether P is different from NP.

Later on in the lectures, we will define the notion of NP-completeness and show
that, while we (today) cannot answer the question ``P vs. NP'', we can identify
a set of problems, the NP-complete problems, that capture the complexity of all
the problems in NP\@. This kind of result relies heavily on the notion of
reduction.

\subsection{Definitions}

We already saw that the reducibility of a problem to
another need not rely on similarity between the two problems. It is
possible to prove reducibility results between two \emph{a priori} unrelated problems.
The goal of this exercise is to show you a first reduction of an NP-complete
problem to a problem in NP that, at first sight, might not seem intuitive.

Let us begin with a few definitions
\begin{definition}
	A boolean formula \(\phi\) over \(n\) variables \(x_1,x_2,\ldots,x_n\) is in
	conjunctive normal form if \(\phi\) can be
	written as a conjunction of \(m\) clauses \(c_1 \land c_2 \land \cdots \land
	c_m\) where each clause \(c_j\) can be written as a disjunction of
	\(k_j\) literals
	\((l_{j1} \lor l_{j2} \lor \cdots \lor l_{jk_j})\) where each literal is
	either a variable \(x_i\) or the negation of a variable \(\lnot x_i\).
	The language CNF is the language of boolean formulae in conjunctive normal
	form.
\end{definition}

The language CNF-SAT is defined as
\begin{definition}
	CNF-SAT \(= \{ \langle \phi \rangle \st \phi \in\) CNF and there exists an
	assignment \(\in \{0,1\}^n\) on the variables \(x_1,x_2,\ldots,x_n\) that
	makes the formula evaluate to true. \(\}\)
\end{definition}

The language 3SAT is a subset of CNF-SAT
\begin{definition}
	3SAT \(= \{ \langle \phi \rangle \st \phi \in \text{CNF-SAT and has
	exactly 3 literals per clause.}\}\)
\end{definition}

The language CLIQUE is the following
\begin{definition}
	CLIQUE \(= \{ \langle G , k \rangle \st \text{ \(G\) is a graph with a clique
	of size \(k\).}\}\)
\end{definition}

\subsection{Statement and proof}

We now prove the following
\begin{theorem}
	3SAT \(\le_P\) CLIQUE
\end{theorem}

\subsubsection{Reduction}

Given a boolean formula \(\phi\) in conjunctive normal form with exactly three
literals per clause, we construct graph \(G = (V,E)\) and then prove that \(G\)
has a clique of size \(m\) if and only if \(\phi\) is satisfiable.

We start with an empty graph. For each clause we add a set of three vertices to
the graph, called a triple. The three added vertices correspond to the three
literals of the clause. For each pair of vertices that are not from the same
triple, we add an edge if the two literals do not contradict each other. Note
that we do not add any edge between the vertices of the same triple.

Here is a corresponding \emph{Turing-machine-like} description of the algorithm
\begin{TMachine}{\(f =\) on input \(\langle \phi \rangle\):}
\item[1.] \(V \gets \emptyset\)
\item[2.] \(E \gets \emptyset\)
\item[3.] For each \(c_j = (l_{j1},l_{j2},l_{j3}) \in \phi\):
\item[3.1.] \(T \gets \{v_{j1},v_{j2},v_{j3}\}\)
\item[3.2.] \(V \gets V \cup T\)
\item[3.3.] For each \(v_{jk} \in T\):
\item[3.3.1.] For each \(v_{j'k'} \in V \setminus T\):
\item[3.3.1.1.] \(E \gets E \cup \{\{v_{jk},v_{j'k'}\}\}\) if \(l_{jk} \neq \lnot l_{j'k'}\)
\item[4.] Output \(\langle (V,E) , m \rangle\)
\end{TMachine}

\subsubsection{Complexity}
An important step when reducing an NP-complete problem to an NP problem is to
check that the reduction takes polynomial time and that the new problem is of
polynomial size.

It is not difficult to see that the complexity of the algorithm above is
quadratic in the number of clauses (\(O(m^2)\)). The size of the output graph
(\(\card{V} + \card{E}\)) is also \(O(m^2)\). \(f\) runs thus in polynomial
time and produces a graph of polynomial size.

\subsubsection{Correctness}

We need to prove that \(\langle \phi \rangle \in \text{3SAT} \iff f(\langle \phi \rangle) \in
\text{CLIQUE}\). We prove the two implications of the equivalence.

\paragraph{(\(\implies\))} If \(\langle\phi\rangle \in \text{3SAT}\) then, by definition,
there exists an assignment \((x_1', x_2', \ldots, x_n') \in \{0,1\}^n\) such
that for each clause there is at least one literal \(l_{jz_j}\) that evaluates
to true. In other words, \(\exists l_{1z_1},l_{2z_2},\ldots,l_{mz_m}\) such
that no two literals \(l_{az_a},l_{bz_b}\) are contradictory, that is,
\(l_{az_a} \neq \lnot l_{bz_b} \Forall a,b \in [m]\). The corresponding
vertices in the graph \(G\) of \(f(\langle \phi \rangle) = \langle G,m
\rangle\)
form a \(m\)-clique, that
is, the graph induced by \(v_{1z_1},v_{2z_2},\ldots,v_{mz_m}\) must be
complete. This holds because, by construction, an edge can only be missing if two
vertices represent literals of the same clause, which is not the case here
since we have selected exactly one literal per clause, \textbf{or} if two
vertices represent contradictory literals, which is not
possible either. Hence, \(f(\langle \phi \rangle) \in \text{CLIQUE}\).

\paragraph{(\(\impliedby\))} If \(f(\langle \phi \rangle) = \langle G, m \rangle \in \text{CLIQUE}\),
then, by definition, \(G\) must contain a clique of size \(m\). By
construction of \(G\), this clique cannot contain more than one vertex per
triple, since there is no edge between two vertices of the same triple. Since
there are \(m\) triples, a clique of size \(m\) must thus contain exactly one
vertex per triple. Let us name those vertices
\(v_{1z_1},v_{2z_2},\ldots,v_{mz_m}\), they correspond to literals
\(l_{1z_1},l_{2z_2},\ldots,l_{mz_m}\) covering all the clauses of \(\phi\).
By construction of \(G\), two such literals cannot be contradictory because
their corresponding vertices, being part of the same clique, have an edge
between them. Hence, \(\langle \phi \rangle \in\) 3SAT.

\subsection{About Karp reductions}

What we just did is called a Karp reduction. Given two languages \(A\)
and \(B\), an algorithm for deciding \(B\), and a polynomial-time
mapping procedure \(f\) such that
\(a \in A\) if and only if \(f(a) \in B\), we can decide whether \(a \in A\) using
the following algorithm
\begin{TMachine}{On input \(a\):}
\item[1.] Translate instance \(a \in A\) to the corresponding instance \(f(a) \in
	B\) in polynomial time.
\item[2.] Answer the same as the algorithm for \(B\) run on \(f(a)\).
\end{TMachine}
Moreover, if the algorithm for \(B\) runs in polynomial time so does this
algorithm.

\subsection{CLIQUE is NP-complete}

During the next lecture, we will define the notion of NP-completeness
\begin{definition}
	A language \(B\) is NP-complete if it satisfies two conditions:
	\begin{enumerate}
		\item \(B\) is in NP, and
		\item every \(A\) in NP is polynomial-time reducible to \(B\).
	\end{enumerate}
\end{definition}
and see that
\begin{theorem}
	CNF-SAT is NP-complete
\end{theorem}
and that
\begin{theorem}
	CNF-SAT \(\le_P\) 3SAT.
\end{theorem}
Those two theorems together with the one we just proved and the fact that
CLIQUE is in NP\footnote{Left as exercise for the reader}
imply the following corollary
\begin{corollary}
	CLIQUE is NP-complete.
\end{corollary}
