\section{Exercises 7.1 and 7.2\footnote{Exercises from the reference book: Sipser M.,
\emph{Introduction to the Theory of Computation}, 3rd edition (2013).}}
Those exercises are simple to understand once you have understood the definitions for the
\emph{Big-O} and \emph{Small-O} notations\footnote{We saw these definitions
during last lecture and it is explained at the beginning of Chapter 7.}.

\subsection{Exercise 7.1}
As a reminder
\begin{definition}
	\(f(n) = O(g(n)) \iff \exists c, n_0\) such that \(f(n) \le c g(n)\) for
	all \(n > n_0\).
\end{definition}

\paragraph{(a)}
\(2n = O(n)\)
is true because \(c = 2\) and \(n_0 = 0\) verify the definition.

\paragraph{(b)}
\(n^2 = O(n)\)
is false because for any \(c \ge 1\) we have that for all \(n > c\), \(n^2 > cn\).

\paragraph{(c)}
\(n^2 = O(n \log^2 n)\)
is false because $n$ grows faster than $\log^2 n$ (hint: check the derivatives).

\paragraph{(d)}
\(n \log n = O(n^2)\)
is true because $\log n$ grows slower than $n$ (hint: check the derivatives).
For instance, \(c = 1\) and \(n_0 = 0\) verify the definition.

\paragraph{(e)}
\(3^n = 2^{O(n)}\)
is true because \(c = \log_2 3\) and \(n_0 = 0\) verify the definition.

\paragraph{(f)}
\(2^{2^n} = O(2^{2^n})\)
is true because \(c = 1\) and \(n_0 = 0\) verify the definition. Note that we
always have \(f(n) = O(f(n))\).

\subsection{Exercise 7.2}
As a reminder
\begin{definition}
	\(f(n) = o(g(n)) \iff \lim_{n \to \infty} \frac{f(n)}{g(n)} = 0\).
\end{definition}

\paragraph{(a)}
\(n = o(2n)\) is false because \(\lim_{n \to \infty} \frac{n}{2n} = \lim_{n \to
\infty} \frac{1}{2} = \frac{1}{2} \neq 0\). Note that we always have \(f(n)
\neq o(k f(n))\) for all constant \(k\).

\paragraph{(b)}
\(2n = o(n^2)\) is true because \(\lim_{n \to \infty} \frac{2n}{n^2} =
\lim_{n \to \infty} \frac{2}{n} = 0\)

\paragraph{(c)}
\(2^n = o(3^n)\) is true because \(\lim_{n \to \infty} \frac{2^n}{3^n} =
\lim_{n \to \infty} \mleft(\frac{2}{3}\mright)^n = 0\).

\paragraph{(d)}
\(1 = o(n)\) is true because \(\lim_{n \to \infty} \frac{1}{n} = 0\).

\paragraph{(e)}
\(n = o(\log n)\) is false because \(\lim_{n \to \infty} \frac{n}{\log n} =
\infty \neq 0\).

\paragraph{(f)}
\(1 = o(\sfrac{1}{n})\) is false because \(\lim_{n \to \infty}
\frac{1}{\sfrac{1}{n}} = \lim_{n \to \infty} n = \infty \neq 0\).
