\section{Exercise 5.31}
We are given the following conjecture
\begin{conjecture}[Collatz's conjecture]\label{collatz}
	Let\footnote{\(\N = \{1,2,\ldots\}\), that is, the set of all natural
		numbers except \(0\).}
\begin{displaymath}
f:\N \to \N:f(x) = \begin{cases}
	3 x + 1 & \text{if \(x\) is odd}\\
	x / 2 & \text{if \(x\) is even}.
\end{cases}
\end{displaymath}
For all \(x \in \N\) the set \(\{x,f(x),f(f(x)),\ldots\}\) contains \(1\).
\end{conjecture}
A conjecture is a statement
for which we do not have a proof that it holds or does not hold.
We want to show that if there existed a decider \(H\) for \Atm{}
then we could build a
\TM{} that would compute a proof of~\ref{collatz} for us.

First, let us define \TM{} \(C\) that recognizes
the subset of \(\N\)
for which the conjecture holds, that is, \(\{x \in \N \st 1 \in
\{x,f(x),f(f(x)),\ldots\} \}\).

\begin{TMachine}{\(C =\) on input \(\langle x \rangle\):}
\item[1.] While \(x \neq 1\): \(x \gets f(x)\)
\item[2.] \(1 \in \{x,f(x),f(f(x)),\ldots\}\) so \accept.
\end{TMachine}

To be able to easily combine\footnote{Alternatively, one can use a mapping from
\(\Sigma\) to \(\N\).} this \TM{} with others, we will define
a \TM{} \(C_{\Sigma^*}\) so that~\ref{collatz} holds if and only if
\(L(C_{\Sigma^*}) =
		\Sigma^*\). Here we take \(\Sigma = \{0,1\}\), the binary alphabet.

\begin{TMachine}{\(C_{\Sigma^*} =\) on input \(w\):}
\item[1.] If\footnote{This can be checked in a finite amount of time.} \(w \not\in
\{\texttt{1},\texttt{10},\texttt{11},\texttt{100},\texttt{101},\texttt{110},\texttt{111},\ldots\}\),
that is, \(w\) is not\footnote{For example, \(w
	= \epsilon\).} the binary representation \(\langle x \rangle\)
		of a number \(x \in \N\), \accept.
	\item[2.] While \(x \neq 1\): \(x \gets f(x)\)
	\item[3.] \(1 \in \{x,f(x),f(f(x)),\ldots\}\) so \accept.
\end{TMachine}

Using \(H\) we build \(R_{\neq \Sigma^*}\), a recognizer for the language
\(L(R_{\neq \Sigma^*}) = \{\langle M\rangle\st L(M) \neq \Sigma^*\}\).

\begin{TMachine}{\(R_{\neq \Sigma^*} =\) on input \(\langle M \rangle\):}
	\item[1.] For each \(w \in \Sigma^*\) in lexicographic order:
	\item[1.1.] Run \(H\) on \(\langle M , w \rangle\).
	\item[1.2.] If \(H\) rejects, \accept.
\end{TMachine}

Finally, we run \(H\) on \(\langle R_{\neq \Sigma^*}, \langle C_{\Sigma^*} \rangle
		\rangle\) to check whether~\ref{collatz} holds. If \(H\) rejects then
		\(L(C_{\Sigma^*}) = \Sigma^*\) and hence the conjecture holds, otherwise the
		conjecture does not hold.
