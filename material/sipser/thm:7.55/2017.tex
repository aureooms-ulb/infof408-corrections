\section{\texorpdfstring{UHAMPATH is NP-complete\footnote{%
Theorem 7.55 from the reference book: Sipser M.,
\emph{Introduction to the Theory of Computation}, 3rd edition
(2013).}}{UHAMPATH is NP-complete}}

Last lecture we showed that
\begin{theorem}
	HAMPATH is NP-complete.
\end{theorem}
We now want to show that the undirected version of HAMPATH is NP-complete too
\begin{definition}
	UHAMPATH \(= \{\langle G, s, t\rangle \st G\) is an undirected
	graph and has an hamiltonian path from vertex \(s\) to vertex
	\(t\}\).
\end{definition}
We prove the following
\begin{theorem}
	UHAMPATH is NP-complete.
\end{theorem}
\begin{proof}
	Clearly, UHAMPATH is in NP\@. For any instance \(\langle G,s,t\rangle\)
	any sequence of vertices that is a
	hamiltonian path from \(s\) to \(t\) in \(G\) can be used as a certificate
	of polynomial size.
	Given an instance \(\langle G, s, t \rangle\) of the HAMPATH problem, we
	build an instance \(\langle G', s^{out}, t^{in}\rangle\) of the UHAMPATH
	problem as follows. Start with an empty undirected graph \(G'\). For the
	vertices \(s\) and \(t\) of \(G\) add the vertices \(s^{out}\) and
	\(t^{in}\) to \(G'\). For each vertex \(v\) other than \(s\) and \(t\) add
	vertices \(v^{in}, v^{mid}\) and \(v^{out}\) to \(G'\). For each triple
	\(v^{in},v^{mid},v^{out}\) of \(G'\) add the undirected edges \(\{v^{in},
	v^{mid}\}\) and \(\{v^{mid},v^{out}\}\) to \(G'\).
	For each directed
	edge \((u,v)\) with \(u \neq t\) and \(v \neq s\) of \(G\) add an
	undirected
	edge \(\{u^{out},v^{in}\}\) to \(G'\).

	Now we prove that \(\langle G, s, t \rangle \in\) HAMPATH if and only if
	\(\langle G', s^{out}, t^{in}\rangle \in\) UHAMPATH\@. A hamiltonian path from \(s\) to
	\(t\) in \(G\) has the following form
	\begin{displaymath}
		s, u_{i_1}, u_{i_2}, \ldots, u_{i_{n-2}}, t
	\end{displaymath}
	which corresponds to the hamiltonian path
	\begin{displaymath}
		s^{out}, u^{in}_{i_1}, u^{mid}_{i_1}, u^{out}_{i_1}, u^{in}_{i_2},
		u^{mid}_{i_2}, u^{out}_{i_2}, \ldots, u^{in}_{i_{n-2}}, u^{mid}_{i_{n-2}},
		u^{out}_{i_{n-2}}, t^{in}
	\end{displaymath}
	in \(G'\). For the other direction of the equivalence we show that a
	hamiltonian path from \(s^{out}\) to \(t^{in}\) in \(G'\) can only have the
	form given above. Such a hamiltonian path must start with \(s^{out}\) and
	since there is no \(s^{mid}\) in \(G'\) we are only able to reach a vertex
	of the type \(u^{in}\). Once in \(u^{in}\) we can only go to \(u^{mid}\).
	This is because the only vertices incident to \(u^{mid}\) are \(u^{in}\) and
	\(u^{out}\). If we do not go to \(u^{mid}\) just after \(u^{in}\),
	the only way to reach \(u^{mid}\) will be through \(u^{out}\) and there
	will be then no way to escape from \(u^{mid}\). Once we are in \(u^{mid}\)
	we continue to \(u^{out}\). We are now in the same configuration as the one
	we started with, except that we removed vertices
	\(s^{out},u^{in},u^{mid}\). To visit all vertices, we must repeat this
	scheme until there is only two vertices left, that is, \(u^{out}_{n-2}\)
	and \(t^{in}\). Note that the complexity of this reduction is linear in the
	size of the input graph \(G\).
\end{proof}
