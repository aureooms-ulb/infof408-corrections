\documentclass{article}

\usepackage{amsmath}
\usepackage{stackengine,scalerel}
\renewcommand\useanchorwidth{T}
\newcommand\obullet[1]{\ThisStyle{\ensurestackMath{%
  \stackon[1pt]{\SavedStyle#1}{\SavedStyle\kern.6\LMpt\bullet}}}}
\newcommand\ocirc[1]{\ThisStyle{\ensurestackMath{%
  \stackon[1pt]{\SavedStyle#1}{\SavedStyle\kern.6\LMpt\circ}}}}

\newcommand{\blank}{\textvisiblespace}
\newcommand\st{\colon\,}
\begin{document}

I feel my explanations have been a little bit sloppy during the first exercise
session so here are more detailed explanations for the exercises we saw.

\section*{3.10}

\subsection*{Introductory remarks}
In the standard definition of the Turing machine, each step writes a symbol to
the cell pointed by the head. Here, we consider that writing the same symbol as
the one that is already in this cell does not count as ``writing''. We
interpret ``writing to a cell'' as altering the content of the cell.

At the beginning of the execution of the Turing machine, the input is
present on the tape and this is counts as zero write operations.

If some set/combinations/thing is of finite size and does not depend on the input
we can encode all its different ``elements'' in the states of the Turing machine.

We call the simulated Turing machine (the one we want to simulate) \(M\) and we
call the simulation Turing machine (the one used to simulate \(M\)) \(S\).

When we say that we add new symbols (\texttt{x}, \texttt{\#}) we suppose they
are not already part of some used alphabet and this is without loss of
generality.
Similarily, we make the blank symbol of the simulated Turing machine and the
simulation Turing machine different. The blank symbol of \(M\)
is \texttt{\obullet{\blank}} while the blank symbol of \(S\)
is \texttt{\blank}. In general you can consider that the tape alphabet of \(M\)
does not include any of the symbols we will add to make \(S\) work.
For example, we can use one color for the tape alphabet of \(M\) and
another color for symbols we add. Here we will consider that all symbols used
by \(M\) are dotted (like \texttt{\obullet{\blank}}) while the symbols that
have a special meaning for \(S\) are not (like \texttt{\blank}).

\subsection*{Using two writes per cell}
The idea is to simulate each step of the simulated
Turing machine by copying all ``ẁorking'' cells to new (unused) cells
while applying the transition function to the cell pointed by the head.

We use a copy procedure that copies each character after
the other by doing ``zig-zags'' between the cells to be copied and the destination
cells. In order to remember which cells were already copied we cross those
cells off during the procedure.

We implement the cross off operation as
writing a special character (for example, \texttt{x}) in the concerned cell.
This operation counts thus as one write operation. The cost of writing the
symbol of the copied cell to the desitination also counts as a write. A cell
will thus be written to at most twice (once as a destination cell, and then
once as a copied cell).

In order to detect when the copy procedure has copied all cells\footnote{%
This fixes an issue of the write-once machine someone pointed out during the
exercise session. Relying on the presence of blank symbols as I explained
during the exercise session will not work. I apologize for the misleading
explanation.}
we add a
special character (for example, \texttt{\#})
at the end of the currently used portion of the tape.
We thus add this special symbol after the last cell being written to by the
copy procedure (and also at the end of the input before beginning the
simulation). The cells containing this special character will only be written to once.

Here are the detailed steps to copy a symbol \texttt{\obullet{\alpha}}:
\begin{enumerate}
	\item Cross off \texttt{\obullet{\alpha}} by writing a \texttt{x}. This
		counts as a single write operation.
	\item Enter a state that remembers that we need to copy symbol
		\texttt{\obullet{\alpha}}, for example, \((q_{\texttt{\alpha}}^1)\). We are allowed
		to do this since alphabets are finite sets (and their size does not
		depend on the input).
	\item While in state \((q_{\texttt{\alpha}}^1)\), move right (without
		altering the current cell) until we reach
		the symbol \texttt{\#}, change then to state
		\((q_{\texttt{\alpha}}^2)\).
	\item While in state \((q_{\texttt{\alpha}}^2)\), move right (without
		altering the current cell) until we reach the symbol \texttt{\blank}.
	\item Replace the \texttt{\blank} by \texttt{\obullet{\alpha}} using a
		single write operation and enter state \((q_{\texttt{\alpha}}^3)\).
	\item While in state \((q_{\texttt{\alpha}}^4)\), move left until we reach
		the symbol \texttt{x}, enter state \((q_{\texttt{\alpha}}^0)\)
		then move right.
\end{enumerate}
A copy of all the cells can be achieved by entering state \((q_{\texttt{\alpha}}^0)\)
then apply the procedure above repeatedly.
The copy stops when in state \((q_{\texttt{\alpha}}^0)\) we read a \texttt{\#}
symbol.

We still need to describe how we can simulate \(M\), that is, remember the
position of the head, update the cell pointed by the head and move the head
according to the transition function of \(M\).

To remember the position of the head of the simulated Turing machine we use
colored symbols (in addition to the color used to differentiate the symbols
used by \(M\) from the symbols having a special meaning for \(S\)).
We do not need to color the first cell at the beginning of
the simulation since we ``know'' the first cell is pointed by the head at the
beginning. Again we can handle this using states since there is a finite number
of possibilities. A simple way to do this is to use special states for the
first steps of \(S\) that will allow us to copy the input with the
appropriate symbols/marks/colors added. Once this is done we can use the
procedures that require the special symbols to be present.

Before copying a cell we need to check whether the next one is the head (is a
colored symbol).
In that case, we have two possibilities:
\begin{enumerate}
\item If the transition function makes the head move left, then
we copy the current cell (the cell to the left of the
head) using the colored version of the symbol, write the symbol designated by
the transition function to the destination cell for the
cell pointed by the head, and then copy the cell to the right of the head
using the normal copy procedure.
\item If the transition function makes the head move right, then
we copy the current cell (the cell to the left of the
head) using the normal copy procedure, write the symbol designated by
the transition function to the destination cell for the
cell pointed by the head, and then copy the cell to the right of the head
using the colored version of the symbol.
\end{enumerate}

We also need to handle two special cases.
If the head was on the leftmost cell and the transition function makes the
head move left then we write the colored version of the symbol designated by
the transition function to the first destination cell and then continue the
copy procedure normally.
If the head was on the last ``used'' cell and the transition function makes
the head move right then we need to write a blank space (\texttt{\obullet{\blank}}) next to the
last destination
cell and put the special
``end-of-the-currently-used-portion-of-the-tape'' symbol (\texttt{\#})
one cell further.

\subsection*{Using a single write per cell}

We transform each cell into two adjacent cells. The left one will be used to
store the symbol and the right one to remember whether the cell has been
crossed off (that is, store the \texttt{x} symbol).

Since all ``odd'' cells contain symbols and ``even'' cells contain cross marks,
it is not difficult to see that we can build a Turing machine that will know
which type of cell the head is currently pointing at (there is a finite number
of types of cells (2) so we can encode this in a finite number of additional
states compared to the write-twice version).

Symbol cells will only be written to once since we do not mark them anymore.
Cross marks cells contain the original blank symbol of the tape until the
corresponding symbol cell has been copied. When the corresponding symbol cell
is copied we write the \texttt{x} symbol in the ``cross-off-mark'' cell. Since a symbol
cell is copied only once, a cross mark cell is written to only once.


\subsection*{Questions}
\subsubsection{Why can we not use a Turing machine with an infinite number of
tapes}
To use this kind of device we must make sure that it is equivalent to a single
tape Turing machine.
A Turing machine with an infinite number of tapes performs an infinite number
of operations at each step. It is not difficult to see that this leads to a
computing model that is more powerful than the Turing machine.

\section*{3.15.a}
A decision problem can be seen as language \(A\) using the following mapping:
\begin{enumerate}
	\item To each instance of the problem is associated a word of some alphabet
		\(\Sigma\) that encodes this instance.
	\item \(A\) is the language (a set of words) containing only
		the words that encode an instance for
		which the decision problem answer is ``yes''. Words encoding an
		instance for which the decision problem is ``no'' of words that do not
		encode an instance (garbage) are excluded of the language.
\end{enumerate}
Deciding the answer to an instance of the problem amounts to determining
whether the word encoding this instance belongs to \(A\).

A Turing machine decides a language if for each word \(w \in \Sigma^{*}\)
of the language alphabet \(\Sigma\)
it either accepts or rejects the word after a finite number of steps. A decider
will never loop forever.

\(A \cup B = {w \st w \in A \text{or} w \in B}\)
is the language that contains all words in \(A\) and all words in \(B\).

If we are given a Turing machine \(M_A\) that decides a language \(A\) and
a turing machine \(M_B\) that decides a language \(B\) we can decide the
language \(A \cup B\) using \(M_{A \cup B}\) defined as follows
(solution given in the textbook):

``On input word \(w\):
\begin{enumerate}
	\item Run \(M_A\) on \(w\). If it accepts, accept.
	\item Run \(M_B\) on \(w\). If it accepts, accept. Otherwise, reject.''
\end{enumerate}
\(M_{A \cup B}\) accepts \(w\) if either \(M_A\) or \(M_B\) accepts it.
If both reject, \(M_{A \cup B}\) rejects.

\section*{3.16.a}

The difference from the previous exercise is that we are given \(M_A\) and
\(M_B\) that are recognizers (not deciders). A recognizer \(M\) will always stop
after a finite number of steps if run on a word that belongs to the language
\(L(M)\) (the language recognized by \(M\)).
However, if the word does not belong to \(L(M)\), \(M\) can either stop and reject or loop forever.

The solution is thus to simulate both machines \(M_A\) and \(M_B\) ``in
parallel'' (instead of in ``sequence'', like above). Solution given in the textbook:

``On input w:
\begin{enumerate}
	\item Run \(M_A\) and \(M_B\) alternately on \(w\) step by step. If either accepts,
accept. If both halt and reject, reject .''
\end{enumerate}

If either \(M_A\) or \(M_B\) accepts \(w\), \(M_{A \cup B}\) accepts \(w\)
because the accepting Turing machine arrives to its
accepting state after a finite number of steps. Note that if both \(M_A\) and
\(M_B\) reject and either of them does so by looping, then \(M_{A \cup B}\)
will loop.

\section*{Other exercices from 3.15 and 3.16}

You can try to solve these exercices by yourself. Here are the definition of
the operators:
\begin{itemize}
\item[concatenation] \( A \times B = { w_a w_b \st w_a \in A \text{and} w_b \in
	B}\), that is, the
	words that are the result of concatenating a word of \(B\) to a word of
	\(A\).
\item[star] \( A* = { \epsilon } \cup { w_1 \cdots w_n \st w_i \in A, n \ge
	1}\), that is,
	words that are the result of concatenating any number of words from \(A\).
\item[complementation] \( \overline{A} = { w \in \Sigma^{*} \st w \not \in A
	}\), that is, the words of the language alphabet that are not in the
	language.
\item[intersection] \( A \cap B = {w \st w \in A \text{and} w \in B}\), that
	is, words that are in both \(A\) and \(B\).
\item[homomorphism] Let \(h \colon \Sigma^{*} \to \Sigma^{*}\)
	be a function such that \(h(w_1 w_2) = h(w_1) h(w_2)\) for any words
	\(w_1\) and \(w_2\). If \(A\) is
	a language, then \(h(A)\) is defined as \({ w \st w = h(x), x \in A}\).
\end{itemize}

It is maybe interesting to understand why complementation is not listed in
3.16.

\end{document}
