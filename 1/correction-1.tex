\documentclass{article}

\begin{document}

I feel I have been a little bit sloppy during the exercise session so here are
more detailed explanations for the exercises we saw.

\section*{3.10}

\subsection*{Introductory remarks}
In the standard definition of the Turing machine, each step writes a symbol to
the cell pointed by the head. Here, we consider that writing the same symbol as
the one that is already in this cell does not count as ``writing''. We
interpret ``writing'' as altering the content of the cell.

At the beginning of the execution of the Turing machine, the input is
present on the tape and this is not counted as a write operation.

When we say that we add new symbols (\texttt{x}, \texttt{\#}) we suppose they
are not already part of some used alphabet and this is without loss of generality.

\subsection*{Using two writes per cell}
The idea is to simulate each step of the simulated
Turing machine by copying all ``ẁorking'' cells to new cells
while applying the transition function to the cell pointed by the head.

We use a copy procedure that copies each character after
the other by doing ``zig-zags'' between the cells to be copied and the destination
cells. In order to remember which cells were already copied we cross those
cells off during the procedure. We implement the cross off operation as
writing a special character (for example, \texttt{x}) in the concerned cell.
This operation counts thus as one write operation. The cost of writing the
symbol of the copied cell to the desitination also counts as a write. A cell
will thus be written to at most twice (once as a destination cell, and then
once as a copied cell).

In order to detect when the copy procedure has copied all cells\footnote{%
This fixes an issue of the write-once machine someone pointed out during the
exercise session.}
we add a
special character (for example, \texttt{\#})
at the end of the currently used portion of the tape.
We thus add this special symbol after the last cell being written to by the
copy procedure (and also at the end of the input before beginning the
simulation). The cells containing this special character will only be written to once.

To remember the position of the head of the simulated Turing machine we use
colored symbols (we do not need to color the first cell at the beginning of
the simulation since we ``know'' the first cell is pointed by the head at the
beginning).
Before copying a cell we need to check whether the next one is the head (is a
colored symbol).
In that case, we have two possibilities:
\begin{enumerate}
\item If the transition function makes the head move left, then
we copy the current cell (the cell to the left of the
head) using the colored version of the symbol, write the symbol designated by
the transition function to the destination cell for the
cell pointed by the head, and then copy the cell to the right of the head
using the normal copy procedure.
\item If the transition function makes the head move right, then
we copy the current cell (the cell to the left of the
head) using the normal copy procedure, write the symbol designated by
the transition function to the destination cell for the
cell pointed by the head, and then copy the cell to the right of the head
using the colored version of the symbol.
\end{enumerate}

We also need to handle two special cases.
If the head was on the leftmost cell and the transition function makes the
head move left then we write the colored version of the symbol designated by
the transition function to the first destination cell and then continue the
copy procedure normally.
If the head was on the last ``used'' cell and the transition function makes
the head move right then we need to write a colored blank space next to the
last destination
cell and put the special
``end-of-the-currently-used-portion-of-the-tape'' symbol (\texttt{\#})
one cell further.

\subsection*{Using a single write per cell}

We transform each cell into two adjacent cells. The left one will be used to
store the symbol and the right one to remember whether the cell has been
crossed (that is, store the \texttt{x} symbol).

Since all ``odd'' cells contain symbols and ``even'' cells contain cross marks,
it is not difficult to see that we can build a Turing machine that will know
which type of cell the head is currently pointing at (there is a finite number
of types of cells (2) so we can encode this in a finite number of additional
states compared to the write-twice version).

Symbol cells will only be written to once since we do not mark them anymore.
Cross marks cells contain the original blank symbol of the tape until the
corresponding symbol cell has been copied. When the corresponding symbol cell
is copied we write the \texttt{x} symbol in the cross mark cell. Since a symbol
cell is copied only once, a cross mark cell is written to only once.


\subsection*{Questions}
\subsubsection{Why can we not use a Turing machine with an infinite number of
tapes}
To use this kind of device we must make sure that it is equivalent to a single
tape Turing machine.
A Turing machine with an infinite number of tapes performs an infinite number
of operations at each step. It is not difficult to see that this leads to a
computing model that is more powerful than the Turing machine.
\section*{3.15.a}
\section*{3.16.a}
\end{document}
